\documentclass{nlctdoc}

\usepackage[colorlinks,
            bookmarks,
            bookmarksopen,
            hyperindex=false,
            pdfauthor={Nicola L.C. Talbot},
            pdftitle={datetime.sty: A Date and Time Package},
            pdfkeywords={date,time,LaTeX}]{hyperref}

\usepackage{creatdtx}

\begin{document}
\RecordChanges
\OnlyDescription

\title{datetime.sty v2.58: 
Formatting Current Date and Time}
 \author{Nicola L. C. Talbot\\[10pt]
 School of Computing Sciences\\
 University of East Anglia\\
 Norwich.  NR4 7TJ.\\
 United Kingdom.\\
 \url{http://theoval.cmp.uea.ac.uk/~nlct/}}
 \date{2010-09-21}
 \maketitle
 \tableofcontents
 \section{Introduction}
\changes{1.0}{2000/08/08}{First release}
\changes{1.01}{2000/09/18}{Documentation added}
The \styfmt{datetime} package is a \LaTeXe\ package that 
provides various different formats for \ics{today},
and provides commands for displaying the current time.  
If you only want the 
time commands but not the date changing commands, you can pass 
the option \pkgopt{nodate} to the package.
\changes{2.3}{2004/05/01}{nodate package option added}

\changes{2.41}{2004/10/22}{split package into two files: 
datetime.sty and fmtcount.sty}
Since version 2.4, the \styfmt{datetime} package has been 
separated into two packages: \styfmt{datetime} and 
\sty{fmtcount}.  When I originally created this package, 
I defined the commands, \ics{ordinal} etc which could be used 
in the definition of \cs{today}.  Since then, I have extended 
the number of commands available that can be used to display the 
value of a \LaTeX\ counter, however it seems more appropriate to 
define all these counter-related commands in a separate package. 
The \sty{fmtcount} package is now distributed separately 
from the \styfmt{datetime} package, and will also need to 
be installed.

\changes{2.42}{2004/11/01}{made package compatible with babel}
As from version 2.42, the \styfmt{datetime} package is now 
compatible with \sty{babel}, however you must load the 
\styfmt{datetime} package \emph{after} the \sty{babel} package.  
For example:
\begin{verbatim}
\usepackage[francais]{babel}
\usepackage{datetime}
\end{verbatim}

 \section{Predefined Date Formats}
 There are various declarations that change the effect of 
\ics{today}.  The change can be localised by placing the 
declaration within a group.

As from version 2.43, the numerical date formats (such as 
\ics{ddmmyyyydate}) use the command
\begin{definition}[\DescribeMacro{\dateseparator}]
\cs{dateseparator}
\end{definition}
to separate the numbers.  So, for example, if you want to hyphens
instead of slashes, you can do:
\begin{verbatim}
\renewcommand{\dateseparator}{-}
\end{verbatim}

\subsection{ISO format}

\begin{definition}[\DescribeMacro{\yyyymmdddate}]
\cs{yyyymmdddate}
\end{definition}
This declaration will redefine \ics{today} to produce the current
date displayed in the form 2000/03/08. (You can redefine
\ics{dateseparator} to \texttt{-} to change to 2000-03-08.)

\subsection{\texorpdfstring{\meta{Day} \meta{Month} \meta{Year}}{Day Month Year} formats}

\begin{definition}[\DescribeMacro{\longdate}]
\cs{longdate}
\end{definition}
This declaration will redefine \ics{today} to produce the current
date displayed in the form Wednesday 8\textsuperscript{th} March,
2000 if the package option \pkgopt{dayofweek} is used, or
8\textsuperscript{th} March, 2000 if the package option
\pkgopt{nodayofweek} is used.

\begin{definition}[\DescribeMacro{\shortdate}]
\cs{shortdate}
\end{definition}
This declaration will redefine \ics{today} to produce the current
date displayed in the form Wed 8\textsuperscript{th} Mar, 2000 if
the package option \pkgopt{dayofweek} is used, or
8\textsuperscript{th} Mar, 2000 if the package option
\pkgopt{nodayofweek} is used.

\begin{definition}[\DescribeMacro{\ddmmyyyydate}]
\cs{ddmmyyyydate}
\end{definition}
This declaration will redefine \ics{today} to produce the current
date displayed in the form 08/03/2000.

\begin{definition}[\DescribeMacro{\dmyyyydate}]
\cs{dmyyyydate}
\end{definition}
This declaration will redefine \ics{today} to produce the current
date displayed in the form 8/3/2000.

\begin{definition}[\DescribeMacro{\ddmmyydate}]
\cs{ddmmyydate}
\end{definition}
This declaration will redefine \ics{today} to produce the current
date displayed in the form 08/03/00.

\begin{definition}[\DescribeMacro{\dmyydate}]
\cs{dmyydate}
\end{definition}
This declaration will redefine \ics{today} to produce the current
date displayed in the form 8/3/00.

\begin{definition}[\DescribeMacro{\textdate}]
\cs{textdate}
\end{definition}
This declaration will redefine \ics{today} to produce the current
date displayed in the form: Wednesday the Eighth of March, Two
Thousand if the package option \pkgopt{dayofweek} is used, or Eighth
of March, Two Thousand if the package option \pkgopt{nodayofweek} is
used. Note that \cs{textdate} is defined for use with English, it
won't look right if it is used when another language has been
selected\footnote{in fact, you may get an error from the
\sty{fmtcount} package if you are using a language that it doesn't
support.}.  If you want to define a similar command for another
language, you will first need to check that the \sty{fmtcount}
package supports that language.

\subsection{\texorpdfstring{\meta{Month} \meta{Day} \meta{Year}}{Month Day Year} formats}

\begin{definition}[\DescribeMacro{\usdate}]
\cs{usdate}
\end{definition}
This declaration will redefine \cs{today} to produce the current
date displayed in the form March 8, 2000. (As \TeX\ and \LaTeX\ do
by default.)

\begin{definition}[\DescribeMacro{\mmddyyyydate}]
\cs{mmddyyyydate}
\end{definition}
This declaration will redefine \ics{today} to produce the current
date displayed in the form 03/08/2000.

\begin{definition}[\DescribeMacro{\mdyyyydate}]
\cs{mdyyyydate}
\end{definition}
This declaration will redefine \verb"\today" to produce the current
date displayed in the form 3/8/2000.

\begin{definition}[\DescribeMacro{\mmddyydate}]
\cs{mmddyydate}
\end{definition}
This declaration will redefine \ics{today} to produce the current
date displayed in the form 03/08/00.

\begin{definition}[\DescribeMacro{\mdyydate}]
\cs{mdyydate}
\end{definition}
This declaration will redefine \ics{today} to produce the current
date displayed in the form 3/8/00.

\subsection{Dates defined by \texorpdfstring{\styfmt{babel}}{babel}}

In addition to the above, the declarations \cs{date}\meta{lang} are
available for all languages defined either by calling \sty{babel}
prior to \styfmt{datetime} or by passing the language name as an
option to \styfmt{datetime}.  See~\sectionref{sec:newdate} if you
want to define your own customised date format.

 \section{Time Commands}

The current time is displayed using the command 
\begin{definition}[\DescribeMacro{\currenttime}]
\cs{currenttime}
\end{definition}

A specific time can be displayed using the command
\begin{definition}[\DescribeMacro{\formattime}]
\cs{formattime}\marg{hour}\marg{minute}\marg{second}
\end{definition}
where \meta{hour} is a number from~0 to~23, and \meta{minute} and
\meta{second} are numbers from~0 to~59.

The format can be changed using the declaration
\begin{definition}[\DescribeMacro{\settimeformat}]
\cs{settimeformat}\marg{style}
\end{definition}
where \meta{style} is the name of the
format. Predefined formats are:
\begin{description}
\item[xxivtime] Twenty-four hour time in the form 22:28 (Default)
\item[hhmmsstime] Twenty-four hour time in the form 22:28:00
\item[ampmtime] Twelve hour time in the form 10:28pm
\item[oclock] Displays the current time as a string, e.g.\ 
Twenty-Eight minutes past Ten in the afternoon.
\end{description}

New time formats can be defined using the command:
\begin{definition}[\DescribeMacro{\newtimeformat}]
\cs{newtimeformat}\marg{name}\marg{format}
\end{definition}
where \meta{name} is the name of the new format (used in
\ics{settimeformat}), and \meta{format} is how to format the time.
Within \meta{format} you can use the counters \ctr{HOUR} (number
of hours after midnight), \ctr{MINUTE} (number of minutes past
the hour), \ctr{SECOND} (number of seconds) \ctr{HOURXII}
(number of hours after midnight/midday), \ctr{TOHOUR} (the next
hour) and \ctr{TOMINUTE} (number of minutes to the next hour),
and the corresponding commands: \ics{THEHOUR}, \ics{THEMINUTE},
\ics{THESECOND}, \ics{THEHOURXII}, \ics{THETOHOUR} and
\ics{THETOMINUTE}.

For example, to define a new time format that uses a dot instead of a
colon:
\begin{verbatim}
\newtimeformat{dottime}{\twodigit{\THEHOUR}.\twodigit{\THEMINUTE}}
\end{verbatim}
You then need to switch to this new format before you can use it:
\begin{verbatim}
\settimeformat{dottime}
\currenttime
\end{verbatim}

As from version 2.43, if you only want to change the separator, 
you can simply redefine
\begin{definition}[\DescribeMacro{\timeseparator}]
\cs{timeseparator}
\end{definition}
instead of defining a new time format.  For example:
\begin{verbatim}
\renewcommand{\timeseparator}{.}
\end{verbatim}
The \texttt{xxivtime} format will now work like the \texttt{dottime} 
format defined above.

 \section{Formating Dates}
\begin{definition}[\DescribeMacro{\pdfdate}]
\cs{pdfdate}
\end{definition}
 This command\footnote{thanks to Ulrich Dirr for 
asking about this} prints the date in the format required for
 PDF files, e.g.\ if the date is 1 May 2004 and time is
 22:02, \verb"\pdfdate" will print 20040501220200.  The reason 
this date format is separate from all the others is because the 
other form doesn't get properly expanded by PDF\TeX. (This 
command is defined regardless of whether the package option 
\texttt{nodate} is called.)
Example:
\begin{verbatim}
\pdfinfo{
   /Author (Me)
   /Title (A Sample Document)
   /CreationDate (D:20040501215500)
   /ModDate (D:\pdfdate)
}
\end{verbatim}
\textbf{Note} that PDF\TeX\ introduced the new primitive
\cs{pdfcreationdate} in version 1.30.0, so \cs{pdfdate} isn't
required with newer versions of PDF\TeX. (\cs{pdfcreationdate}
is better than \cs{pdfdate} as it also stores the time zone.)
 
 There are two commands that print the name of the current
 month:
\begin{definition}[\DescribeMacro{\monthname}]
\cs{monthname}\oarg{num}
\end{definition}
prints the current month name in full, 
e.g.\ August, and 
\begin{definition}[\DescribeMacro{\shortmonthname}]
\cs{shortmonthname}\oarg{num}
\end{definition}
prints the abbreviated month name, e.g.\ Aug.  Both \cs{monthname}
and \cs{shortmonthname} take an optional argument (a number from 1
to 12) if the name of a specific month is required.  For example,
\verb"\monthname[6]" will produced the output: June.

 The day of the week is computed using the algorithm documented at
 \url{http://userpages.wittenburg.edu/bshelburne/Comp150/DayOfWeek.htm}.
 This algorithm works for any date between 1\textsuperscript{st} 
Jan, 1901 and 31\textsuperscript{st} Dec, 2099.
 The following macros display the day of week for a given date:

\begin{definition}[\DescribeMacro{\dayofweekname}]
\cs{dayofweekname}\marg{day}\marg{month}\marg{year}
\end{definition}
prints the day of week for the specified date.  For example, 
\begin{verbatim}
\dayofweekname{31}{10}{2002}
\end{verbatim}
will produce the output: Thursday.

\begin{definition}[\DescribeMacro{\shortdayofweekname}]
\cs{shortdayofweekname}\meta{day}\marg{month}\marg{year}
\end{definition}
prints the abbreviated name for the day of week for the specified
date. For example
\begin{verbatim}
\shortdayofweekname{31}{10}{2002}
\end{verbatim}
will produce the output: Thu.

\begin{definition}[\DescribeMacro{\ifshowdow}]
\cs{ifshowdow}
\end{definition}
This \TeX\ conditional can be used to determine whether or not the
option \pkgopt{dayofweek} has been passed to the package. For
example:
\begin{verbatim}
\ifshowdow\dayofweekname{31}{10}{2002} \fi
\end{verbatim}
will only display the day of week if the \pkgopt{dayofweek} option
was passed to \styfmt{datetime}. Alternatively, you can use
David~Carlisle's \sty{ifthen} package:
\begin{verbatim}
\ifthenelse{\boolean{showdow}}{\dayofweekname{31}{10}{2002} }{}
\end{verbatim}

The command
\begin{definition}[\DescribeMacro{\ordinaldate}]
\cs{ordinaldate}\marg{number}
\end{definition}
displays \meta{number} as a date-type ordinal. If the current
language is English, this will simply pass the argument to
\ics{ordinalnum} (defined in the \sty{fmtcount} package), if the
current language is Breton, Welsh or French, a superscript will only
be added if \meta{number} is 1, otherwise only \meta{number} will be
displayed.

The macro\footnote{Note the name change since version 1.1. The
command name was changed from \cs{thedate} to \cs{formatdate} to
avoid a name clash when using the \sty{seminar} class file.}
\begin{definition}[\DescribeMacro{\formatdate}]
\cs{formatdate}\marg{day}\marg{month}\marg{year}
\end{definition}
formats the specified date according to the current format of
\ics{today}\footnote{To be more precise, \ics{today} is defined to
be \ics{formatdate}\{\cs{day}\}\{\cs{month}\}\{\cs{year}\} where
\ics{longdate} etc change the definition of \cs{formatdate}}.
(Arguments must all be integers.) For example, in combination with
\ics{longdate}, the command
\begin{verbatim}
\formatdate{27}{9}{2004}
\end{verbatim}
will produce the output: Monday 27\textsuperscript{th} September,
2004.

You can ensure that a number is displayed with at least two digits
using the command
\begin{definition}[\DescribeMacro{\twodigit}]
\cs{twodigit}\marg{num}
\end{definition}

\section{Defining New Date Formats}\label{sec:newdate}

New date formats can be defined using the command:
\begin{definition}[\DescribeMacro{\newdateformat}]
 \cs{newdateformat}\marg{name}\marg{format}
\end{definition}
where \meta{name} is the name of the new format, and \meta{format}
is how to format the date. Within the argument \meta{format} you can
use the commands \ics{THEDAY}, \ics{THEMONTH} and \ics{THEYEAR} to
represent the relevant day, month and year, or you can use the
counters \ctr{DAY}, \ctr{MONTH} and \ctr{YEAR} if you want to use
\ics{ordinal} etc. Once you have defined the new date format, you
can then switch to it using the declaration \verb'\'\meta{name}
(i.e.\ the name you specified preceded by a backslash), and
subsequent calls to \ics{today} and \ics{formatdate} will use your
new format.

For example, suppose you want to define a new date format called,
say, \texttt{mydate}, that will typeset the date in the form: 
8-3-2002, then you can do:
\begin{verbatim}
\newdateformat{mydate}{\THEDAY-\THEMONTH-\THEYEAR}
\end{verbatim}
\ics{newdateformat} will then define the declaration 
\cs{mydate} which can be used to
switch to your new format. In the following example, 
two new date formats are defined, and they are then
selected to produce two different formats for the current date:
\begin{verbatim}
\newdateformat{dashdate}{%
\twodigit{\THEDAY}-\twodigit{\THEMONTH}-\THEYEAR}

\newdateformat{usvardate}{%
\monthname[\THEMONTH] \ordinal{DAY}, \THEYEAR}

Dash: \dashdate\today.
US: \usvardate\today.
\end{verbatim}
If the current date is, say, 8th March, 2002, the above code will
produce the following: Dash: 08-03-2002.  US: March
8\textsuperscript{th}, 2002.

Note that \ics{THEDAY} etc and \ctr{DAY} etc have no real 
meaning outside \ics{newdateformat} (this is why they 
are in uppercase). Incidentally, the \texttt{dashdate} format
is not really necessary, as you can achieve this format
using:
\begin{verbatim}
\renewcommand{\dateseparator}{-}
\ddmmyyyydate
\end{verbatim}

Another note: in the above code, \ics{ordinal} was
used to illustrate the use of the \ctr{DAY} counter.  It
is better to use \ics{ordinaldate} instead:
\begin{verbatim}
\newdateformat{usvardate}{%
\monthname[\THEMONTH] \ordinaldate{\THEDAY}, \THEYEAR}
\end{verbatim}

\section{Saving Dates}

It is possible to save a date for later use using the command:
\footnote{Thanks to Denis Bitouz\'e for asking about this}
\begin{definition}[\DescribeMacro{\newdate}]
\cs{newdate}\marg{name}\marg{day}\marg{month}\marg{year}
\end{definition}

This date can later be displayed using the same format as that
used by \ics{formatdate} using the command:
\begin{definition}[\DescribeMacro{\displaydate}]
\cs{displaydate}\marg{name}
\end{definition}

Individual elements of the date can be extracted using the
commands:
\begin{definition}[\DescribeMacro{\getdateday}]
\cs{getdateday}\marg{name}
\end{definition}
\begin{definition}[\DescribeMacro{\getdatemonth}]
\cs{getdatemonth}\marg{name}
\end{definition}
\begin{definition}[\DescribeMacro{\getdateyear}]
\cs{getdateyear}\marg{name}
\end{definition}

\section{Predefined Names}

The following commands are defined by the \styfmt{datetime} 
package:

\begin{tabular}{ll}
\bfseries Command Name & \bfseries Default Value\\
\ics{dateseparator} & \verb'/'\\
\ics{timeseparator} & \verb':'\\
\ics{amname} & \texttt{am}\\
\ics{pmname} & \texttt{pm}\\
\ics{amorpmname} & \ics{amname} if morning, otherwise \ics{pmname}\\
\ics{amstring} & \texttt{in the morning}\\
\ics{pmstring} & \texttt{in the afternoon}\\
\ics{amorpmstring} & \ics{amstring} if morning, otherwise 
\ics{pmstring}\\
\ics{halfpast} & \texttt{Half past}\\
\ics{quarterpast} & \texttt{Quarter past}\\
\ics{quarterto} & \texttt{Quarter to}\\
\ics{noon} & \texttt{Noon}\\
\ics{midnight} & \texttt{Midnight}\\
\ics{oclockstring} & \texttt{O'Clock}
\end{tabular}

\section{Package Options}

 The following options may be passed to this package:
\begin{description}
\item[\pkgopt{long}] make \cs{today} produce long date
\item[\pkgopt{short}] make \cs{today} produce short date
\item[\pkgopt{yyyymmdd}] make \cs{today} produce YYYY/MM/DD date
\item[\pkgopt{ddmmyyyy}] make \cs{today} produce DD/MM/YYYY date
\item[\pkgopt{dmyyyy}] make \cs{today} produce D/M/YYYY date
\item[\pkgopt{ddmmyy}] make \cs{today} produce DD/MM/YY date
\item[\pkgopt{dmyy}] make \cs{today} produce D/M/YY date
\item[\pkgopt{text}] make \cs{today} produce text date
\item[\pkgopt{us}] make \cs{today} produce US style date
\item[\pkgopt{mmddyyyy}] make \cs{today} produce MM/DD/YYYY date
\item[\pkgopt{mdyyyy}] make \cs{today} produce M/D/YYYY date
\item[\pkgopt{mmddyy}] make \cs{today} produce MM/DD/YY date
\item[\pkgopt{mdyy}] make \cs{today} produce M/D/YY date
\item[\pkgopt{raise}] make ordinal st,nd,rd,th appear as superscript
\item[\pkgopt{level}] make ordinal st,nd,rd,th appear level with rest of 
text
\item[\pkgopt{dayofweek}] make the day of week appear for \ics{longdate}, 
\ics{shortdate} 
            or \ics{textdate}
\item[\pkgopt{nodayofweek}] don't display the day of week.
\item[\pkgopt{hhmmss}] make \cs{currenttime} produce \texttt{hhmmsstime} 
format
\item[\pkgopt{24hr}] make \cs{currenttime} produce \texttt{xxivtime} 
format
\item[\pkgopt{12hr}] make \cs{currenttime} produce \texttt{ampmtime} 
format
\item[\pkgopt{oclock}] make \cs{currenttime} produce \texttt{oclock} 
format
\item[\pkgopt{nodate}] Don't redefine \cs{today} or define the month or 
day of week commands
           (useful if you only want the time commands) 
\item[\pkgopt{iso}] as \pkgopt{ddmmyyyy} and \pkgopt{hhmmss} but
also sets date separator to \texttt{-}
 and time separator to \texttt{:}
 \end{description}

The default options are: \pkgopt{long}, \pkgopt{raise},
\pkgopt{dayofweek} and \pkgopt{24hr}.

\section{Multilingual Support}

If you use the \sty{ngerman} package, you must use \styfmt{datetime}'s
\pkgopt{ngerman} package option if you want the date displayed
in the same form as \sty{ngerman}. You must also load \sty{ngerman}
\emph{before} you load \styfmt{datetime}. Similarly for the
\sty{german} package.

If you want to use the \sty{babel} package, you must load
it \emph{before} you load the \styfmt{datetime} package. This
is because the \sty{babel} \cs{date}\meta{lang} commands
redefine \ics{today}, whereas the \sty{datetime} package
redefines \cs{today} to use \ics{formatdate}, and the
date formatting commands (such as \ics{longdate}) redefine
\cs{formatdate} rather than \cs{today}. This ensures 
consistent formatting of the dates whether you use \cs{today} or
\cs{formatdate}.  Therefore, the \styfmt{datetime} package
has to redefine all the \cs{date}\meta{lang} commands accordingly.
Thus the multilingual date support is mostly limited to that provided
by \sty{babel}. Additional support, such as the day of
week names and abbreviations, are only supplied for those
languages that I know, or that other people have been able to supply
for me.

As from v2.56, the package options supplied to \styfmt{datetime}
override \sty{babel}'s date format. For example:
\begin{verbatim}
\usepackage[french,spanish]{babel}
\usepackage[ddmmyyyy]{datetime}
\end{verbatim}
will use ddmmyyyy date format regardless of language. Note that
if you use a declaration instead of a package option, for
example:
\begin{verbatim}
\usepackage[french,spanish]{babel}
\usepackage{datetime}
\ddmmyyyydate
\end{verbatim}
the date format will be switched back to \cs{datefrench} or
\cs{datespanish} whenever the language is set. You can use
\begin{definition}[\DescribeMacro{\setdefaultdate}]
\cs{setdefaultdate}\marg{declaration}
\end{definition}
to always use \marg{declaration} whenever the language
is set. For example:
\begin{verbatim}
\setdefaultdate{\ddmmyyyydate}
\end{verbatim}

The commands \ics{monthname} and \ics{shortmonthname},
will produce the month name in the current language.
If you want the month name in a specific language, you
can use the command \cs{monthname}\meta{lang}.
For example, \verb"\monthnamefrench[6]" will produce the output: 
juin. Note that \ics{textdate} is formatted for English dates,
and won't look right if used with another language setting. If you
want a textual date, the \sty{fmtcount} package (which is 
loaded by \styfmt{datetime}) defines some commands which display
a number or ordinal as a word, but it only has very limited 
multilingual support. See the \sty{fmtcount} documentation 
for further details.

There is currently only \emph{limited} multilingual support for 
\ics{dayofweekname} and \ics{shortdayofweekname} (just
English, French, Portuguese, Spanish and German\footnote{thanks
to Uwe Bieling for supplying the German names}). You can add support 
for other languages by defining the commands
\cs{dayofweeknameid}\meta{lang} and 
\cs{shortdayofweeknameid}\meta{lang}. Note that these
commands only take \emph{one} argument which should be
a number from~1 to~7 indicating the day of the week.

You can use the following as templates.  Replace
\texttt{english} with the name of your language (as given
by \ics{languagename}) and replace \texttt{Sunday}
etc as appropriate:
\begin{verbatim}
\providecommand*{\dayofweeknameidenglish}[1]{%
\ifcase#1\relax
\or Sunday%
\or Monday%
\or Tuesday%
\or Wednesday%
\or Thursday%
\or Friday%
\or Saturday%
\fi}
                                                          
\providecommand*{\shortdayofweeknameidenglish}[1]{%
\ifcase#1\relax
\or Sun%
\or Mon%
\or Tue%
\or Wed%
\or Thu%
\or Fri%
\or Sat%
\fi}
\end{verbatim}
If you want them added to future versions of 
\sty{datetime}, please email me the code.

\section{Registers}

\TeX\ provides \ics{day}, \ics{month} and \ics{year} registers. In
addition, \styfmt{datetime} provides the registers: 
\ics{currenthour}, \ics{currentminute} and \ics{currentsecond}. 
Note that old distributions of \TeX\ will always have
\cs{currentsecond} set to zero.

\section{Configuration File}

As from Version 2.4, the \styfmt{datetime} package will read in settings
from the configuration file \texttt{datetime.cfg}, if it exists,
which will override the default package options.  For example,
suppose you prefer a short date without the day of week by default,
you will need to create a file called \texttt{datetime.cfg} that
contains the line:
\begin{verbatim}
\shortdate\showdowfalse
\end{verbatim}
The file \texttt{datetime.cfg} should then go somewhere on the 
\TeX\ path.  Now all you need to do is:
\begin{verbatim}
\usepackage{datetime}
\end{verbatim}
without having to specify the \pkgopt{short} and 
\pkgopt{nodayofweek} options.

You can also use this file to define and set your own date 
styles.  For example, you could create
a configuration file that has the following lines:
\begin{verbatim}
\newdateformat{dashdate}{\twodigit{\THEDAY}-\twodigit{\THEMONTH}-\THEYEAR}
\dashdate
\end{verbatim}
Whenever you use the \styfmt{datetime} package, it will now 
use this format by default.

\section{LaTeX2HTML styles}

\changes{2.43}{2005/02/25}{Added LaTeX2HTML support}%
\changes{2.44}{2005/03/03}{Fixed minor bugs in Perl scripts}
Version 2.43 and above of the \styfmt{datetime} bundle 
supplies the LaTeX2HTML style file \texttt{datetime.perl}.  
This file should be placed in a 
directory searched by LaTeX2HTML.  The following limitations 
apply to the LaTeX2HTML styles:

\begin{itemize}
\item The configuration file \texttt{datetime.cfg}
is currently ignored.  You can however
do:
\begin{verbatim}
\usepackage{datetime}
\html{%%
%% This is file `datetime.sty',
%% generated with the docstrip utility.
%%
%% The original source files were:
%%
%% datetime.dtx  (with options: `datetime.sty,package')
%% 
%%  datetime.dtx
%%  Copyright 2010 Nicola Talbot
%% 
%%  This work may be distributed and/or modified under the
%%  conditions of the LaTeX Project Public License, either version 1.3
%%  of this license of (at your option) any later version.
%%  The latest version of this license is in
%%    http://www.latex-project.org/lppl.txt
%%  and version 1.3 or later is part of all distributions of LaTeX
%%  version 2005/12/01 or later.
%% 
%%  This work has the LPPL maintenance status `maintained'.
%% 
%%  The Current Maintainer of this work is Nicola Talbot.
%% 
%%  This work consists of the files datetime.dtx and datetime.ins and the derived files datetime.sty, dt-american.def, dt-austrian.def, dt-bahasa.def, dt-basque.def, dt-breton.def, dt-british.def, dt-bulgarian.def, dt-catalan.def, dt-croatian.def, dt-czech.def, dt-danish.def, dt-dutch.def, dt-esperanto.def, dt-estonian.def, dt-finnish.def, dt-french.def, dt-galician.def, dt-german.def, dt-greek.def, dt-hebrew.def, dt-icelandic.def, dt-irish.def, dt-italian.def, dt-latin.def, dt-lsorbian.def, dt-magyar.def, dt-naustrian.def, dt-ngerman.def, dt-norsk.def, dt-polish.def, dt-portuges.def, dt-romanian.def, dt-russian.def, dt-samin.def, dt-scottish.def, dt-serbian.def, dt-slovak.def, dt-slovene.def, dt-spanish.def, dt-swedish.def, dt-turkish.def, dt-UKenglish.def, dt-ukraineb.def, dt-USenglish.def, dt-usorbian.def, dt-welsh.def, dt-lang.tex, dt-sampl.tex, datetime.perl.
%% 
%% \CharacterTable
%%  {Upper-case    \A\B\C\D\E\F\G\H\I\J\K\L\M\N\O\P\Q\R\S\T\U\V\W\X\Y\Z
%%   Lower-case    \a\b\c\d\e\f\g\h\i\j\k\l\m\n\o\p\q\r\s\t\u\v\w\x\y\z
%%   Digits        \0\1\2\3\4\5\6\7\8\9
%%   Exclamation   \!     Double quote  \"     Hash (number) \#
%%   Dollar        \$     Percent       \%     Ampersand     \&
%%   Acute accent  \'     Left paren    \(     Right paren   \)
%%   Asterisk      \*     Plus          \+     Comma         \,
%%   Minus         \-     Point         \.     Solidus       \/
%%   Colon         \:     Semicolon     \;     Less than     \<
%%   Equals        \=     Greater than  \>     Question mark \?
%%   Commercial at \@     Left bracket  \[     Backslash     \\
%%   Right bracket \]     Circumflex    \^     Underscore    \_
%%   Grave accent  \`     Left brace    \{     Vertical bar  \|
%%   Right brace   \}     Tilde         \~}
\NeedsTeXFormat{LaTeX2e}
\ProvidesPackage{datetime}[2010/09/21 v2.58 Date Time Package]
\RequirePackage{fmtcount}
\RequirePackage{ifthen}
\newif\if@dt@nodate
\@dt@nodatefalse
\newcommand{\dateseparator}{/}
\newif\if@dt@multilingual
\@ifpackageloaded{babel}{%
\@dt@multilingualtrue}{%
\@ifpackageloaded{ngerman}{%
\@dt@multilingualtrue}{\@dt@multilingualfalse}}
\newcommand*{\ordinaldateenglish}[1]{\ordinalnum{#1}}
\newcommand*{\ordinaldate}[1]{%
\if@dt@multilingual
\@ifundefined{ordinaldate\languagename}{#1}{%
\csname ordinaldate\languagename\endcsname{#1}}%
\else
\ordinalnum{#1}%
\fi}
\newif\ifshowdow
\providecommand*{\formatdate}[3]{}
\newcount\@day
\newcount\@month
\newcount\@year
\DeclareRobustCommand*{\longdate}{%
\renewcommand*{\formatdate}[3]{%
\ifshowdow\dayofweekname{##1}{##2}{##3} \fi
\@day=##1\relax\@month=##2\relax\@year=##3\relax
\ordinaldate{\the\@day}\ \monthname[\@month], \the\@year}}
\DeclareRobustCommand*{\shortdate}{%
\renewcommand*{\formatdate}[3]{%
\ifshowdow\shortdayofweekname{##1}{##2}{##3} \fi
\@day=##1\relax\@month=##2\relax\@year=##3\relax
\ordinaldate{\the\@day}\ \shortmonthname[\@month], \the\@year}}
\let\twodigit\two@digits
\DeclareRobustCommand*{\yyyymmdddate}{%
\renewcommand*{\formatdate}[3]{%
\@day=##1\relax\@month=##2\relax\@year=##3\relax
\the\@year\dateseparator \twodigit\@month\dateseparator
\twodigit\@day}}
\DeclareRobustCommand*{\ddmmyyyydate}{%
\renewcommand*{\formatdate}[3]{%
\@day=##1\relax\@month=##2\relax\@year=##3\relax
\twodigit\@day\dateseparator \twodigit\@month\dateseparator
\the\@year}}
\DeclareRobustCommand*{\dmyyyydate}{%
\renewcommand*{\formatdate}[3]{%
\@day=##1\relax\@month=##2\relax\@year=##3\relax
\the\@day\dateseparator \the\@month\dateseparator \the\@year}}
\DeclareRobustCommand*{\ddmmyydate}{\renewcommand*{\formatdate}[3]{%
\@day=##1\relax\@month=##2\relax\@year=##3\relax
\@dtctr=\@year%
\@modulo{\@dtctr}{100}%
\twodigit\@day\dateseparator \twodigit\@month\dateseparator
\twodigit\@dtctr}}
\DeclareRobustCommand*{\dmyydate}{\renewcommand*{\formatdate}[3]{%
\@day=##1\relax\@month=##2\relax\@year=##3\relax
\@dtctr=\@year%
\@modulo{\@dtctr}{100}%
\the\@day\dateseparator \the\@month\dateseparator \twodigit\@dtctr}}
\DeclareRobustCommand*{\textdate}{%
\renewcommand*{\formatdate}[3]{%
\ifshowdow\dayofweekname{##1}{##2}{##3} the \fi
\@day=##1\relax\@month=##2\relax\@year=##3\relax
\Ordinalstringnum{\@day}\ of \monthname[\@month],
\Numberstringnum{\@year}%
}}
\DeclareRobustCommand*{\usdate}{%
\renewcommand*{\formatdate}[3]{%
\@day=##1\relax\@month=##2\relax\@year=##3\relax
\monthname[\@month]\ \the\@day, \the\@year}}
\DeclareRobustCommand*{\mmddyyyydate}{%
\renewcommand*{\formatdate}[3]{%
\@day=##1\relax\@month=##2\relax\@year=##3\relax
\twodigit\@month\dateseparator \twodigit\@day\dateseparator
\the\@year}}
\DeclareRobustCommand*{\mdyyyydate}{%
\renewcommand*{\formatdate}[3]{%
\@day=##1\relax\@month=##2\relax\@year=##3\relax
\the\@month\dateseparator \the\@day\dateseparator \the\@year}}
\DeclareRobustCommand*{\mmddyydate}{\renewcommand*{\formatdate}[3]{%
\@day=##1\relax\@month=##2\relax\@year=##3\relax
\@dtctr=\@year%
\@modulo{\@dtctr}{100}%
\twodigit\@month\dateseparator \twodigit\@day\dateseparator
\twodigit\@dtctr}}
\DeclareRobustCommand*{\mdyydate}{\renewcommand*{\formatdate}[3]{%
\@day=##1\relax\@month=##2\relax\@year=##3\relax
\@dtctr=\@year%
\@modulo{\@dtctr}{100}%
\the\@month\dateseparator \the\@day\dateseparator \twodigit\@dtctr}}
\newcommand*{\newdate}[4]{%
\@ifundefined{date@#1@y}{%
\@namedef{date@#1@d}{#2}%
\@namedef{date@#1@m}{#3}%
\@namedef{date@#1@y}{#4}}{%
\PackageError{datetime}{Date `#1' already defined}{}}}
\newcommand*{\getdateyear}[1]{%
\@ifundefined{date@#1@y}{%
\PackageError{datetime}{Date `#1' not defined}{}}{%
\csname date@#1@y\endcsname}}
\newcommand*{\getdatemonth}[1]{%
\@ifundefined{date@#1@m}{%
\PackageError{datetime}{Date `#1' not defined}{}}{%
\csname date@#1@m\endcsname}}
\newcommand{\getdateday}[1]{%
\@ifundefined{date@#1@d}{%
\PackageError{datetime}{Date `#1' not defined}{}}{%
\csname date@#1@d\endcsname}}
\newcommand*{\displaydate}[1]{%
\@ifundefined{date@#1@y}{%
\PackageError{datetime}{Date `#1' not defined}{}}{%
\formatdate{\csname date@#1@d\endcsname}{%
\csname date@#1@m\endcsname}{%
\csname date@#1@y\endcsname}}}
\longdate
\showdowtrue
\providecommand*\THEDAY{\the\@day}
\providecommand*\THEMONTH{\the\@month}
\providecommand*\THEYEAR{\the\@year}
\let\c@DAY=\@day
\let\c@MONTH=\@month
\let\c@YEAR=\@year
\providecommand{\newdateformat}[2]{%
\@ifundefined{#1}{%
\expandafter\DeclareRobustCommand\csname#1\endcsname{%
\renewcommand{\formatdate}{\dateformat{#2}}}}{%
\PackageError{datetime}{Can't create new date format, command
\textbackslash#1 already defined}{You will need to
give your new date format a different name}}}
\DeclareRobustCommand*{\currenttime}{%
  \formattime{\currenthour}{\currentminute}{\currentsecond}%
}
\newcommand*{\formattime}[3]{\protect\@formattime{#1}{#2}{#3}}
\newcommand*{\@formattime}[3]{%
  \csname timeformat@xxivtime\endcsname{#1}{#2}{#3}}
\newcommand*{\timeseparator}{:}
\providecommand*{\settimeformat}[1]{%
  \@ifundefined{timeformat@#1}{%
    \PackageError{datetime}{Unknown time format `#1'}{}%
  }{%
    \renewcommand*{\@formattime}[3]{%
      \csname timeformat@#1\endcsname{##1}{##2}{##3}}%
  }%
}
\newcount\c@HOUR
\newcount\c@HOURXII
\newcount\c@MINUTE
\newcount\c@TOHOUR
\newcount\c@TOMINUTE
\newcount\c@SECOND
\def\THEHOUR{\the\c@HOUR}
\def\THEHOURXII{\the\c@HOURXII}
\def\THEMINUTE{\the\c@MINUTE}
\def\THETOHOUR{\the\c@TOHOUR}
\def\THETOMINUTE{\the\c@TOMINUTE}
\def\THESECOND{\the\c@SECOND}
\newcount\currenthour
\newcount\currentminute
\newcount\currentsecond
{\catcode`\D=12\relax
  \gdef\@dt@getdate D:#1#2#3#4#5#6#7#8{\@dt@gettime}%
}
\@ifundefined{pdfcreationdate}{%
  \currenthour=\time\relax
  \divide\currenthour by 60\relax
  \currentminute=\time\relax
  \@modulo{\currentminute}{60}%
  \currentsecond=0\relax
}{%
  \def\@dt@gettime#1#2#3#4#5#6#7\relax{%
    \currenthour=#1#2\relax
    \currentminute=#3#4\relax
    \currentsecond=#5#6\relax}%
  \expandafter\@dt@getdate\pdfcreationdate\relax
}
\providecommand*{\newtimeformat}[2]{%
\@ifundefined{#1}{%
\expandafter\def\csname#1\endcsname{%
  \csname timeformat@#1\endcsname{\currenthour}{\currentminute}%
  {\currentsecond}%
}%
\expandafter\def\csname timeformat@#1\endcsname##1##2##3{%
\c@HOUR=##1%
\c@HOURXII=\c@HOUR
\ifnum\c@HOURXII>12
\advance\c@HOURXII by -12\relax
\fi
\c@MINUTE=##2%
\c@TOHOUR=\c@HOURXII
\advance\c@TOHOUR by 1\relax
\@modulo{\c@TOHOUR}{12}%
\c@TOMINUTE=\c@MINUTE
\advance\c@TOMINUTE by -60\relax
\multiply\c@TOMINUTE by -1\relax
\c@SECOND=##3\relax
#2\relax
}}{%
\PackageError{datetime}{Command \textbackslash#1  already defined}{%
You can't create a new time format called "#1" as the command
\textbackslash#1 already exists}}}
\newtimeformat{xxivtime}{%
\twodigit\THEHOUR\timeseparator\twodigit\THEMINUTE}
\newtimeformat{hhmmsstime}{%
\twodigit\THEHOUR\timeseparator\twodigit\THEMINUTE\timeseparator
\twodigit\THESECOND}
\newtimeformat{ampmtime}{%
\ifthenelse{\value{HOUR}=0}{12}{\THEHOURXII}\timeseparator
\twodigit\THEMINUTE
\ifthenelse{\value{HOUR}<12}{\amname}{%
\ifthenelse{\value{HOUR}=12}{ \noon}{\pmname}}}
\newcommand*{\hourstring}[1]{%
\ifthenelse{\value{#1}=0}{\midnight}{\Numberstring{#1}}}
\newtimeformat{oclock}{%
  \ifthenelse{\(\value{HOUR}=0 \OR \value{HOUR}=12 \OR\value{HOUR}=24\)\AND
 \value{MINUTE}=0}{%
\ifthenelse{\value{HOUR}=12}{\noon}{\midnight}}{%
\ifthenelse{\value{MINUTE}=0}{%
\Numberstring{HOUR} \oclockstring}{%
\ifthenelse{\value{MINUTE}=15}{%
\quarterpast\ \hourstring{HOUR}}{%
\ifthenelse{\value{MINUTE}=30}{%
\halfpast\ \hourstring{HOUR}}{%
\ifthenelse{\value{MINUTE}=45}{%
\quarterto\ \hourstring{TOHOUR}}{%
\ifthenelse{\value{MINUTE}<30}{%
\Numberstring{MINUTE}\ \ifthenelse{\value{MINUTE}=1}{minute}{minutes}
past \hourstring{HOURXII}}{%
\Numberstring{TOMINUTE}\ \ifthenelse{%
\value{TOMINUTE}=1}{minute}{minutes} to \hourstring{TOHOUR}}}}}}%
\ifthenelse{\value{HOUR}<12}{%
\ifthenelse{\value{HOUR}=0}{}{\ \amstring}}{%
\ifthenelse{\value{TOHOUR}=0}{}{\ \pmstring}}}}
\providecommand*{\amname}{am}
\providecommand*{\pmname}{pm}
\providecommand*{\amorpmname}{%
\ifthenelse{\value{HOUR}>12}{\pmname}{\amname}}
\providecommand*{\amstring}{in the morning}
\providecommand*{\pmstring}{in the afternoon}
\providecommand*{\amorpmstring}{%
\ifthenelse{\value{HOUR}>12}{\pmstring}{\amstring}}
\providecommand*{\halfpast}{Half past}
\providecommand*{\quarterpast}{Quarter past}
\providecommand*{\quarterto}{Quarter to}
\providecommand*{\noon}{Noon}
\providecommand*{\midnight}{Midnight}
\providecommand*{\oclockstring}{O'Clock}
\InputIfFileExists{datetime.cfg}{\PackageInfo{datetime}{%
Loading local datetime configurations}}{%
\PackageInfo{datetime}{No datetime.cfg file found, using default
settings}}
\@ifpackageloaded{babel}{%
  \newcommand*{\dt@addtoextras}[1]{%
    \@ifundefined{extrasenglish}{}{%
      \addto\extrasenglish{#1}%
    }%
    \@ifundefined{extrasamerican}{}{%
      \addto\extrasamerican{#1}%
    }%
    \@ifundefined{extrasaustrian}{}{%
      \addto\extrasaustrian{#1}%
    }%
    \@ifundefined{extrasbahasa}{}{%
      \addto\extrasbahasa{#1}%
    }%
    \@ifundefined{extrasbreton}{}{%
      \addto\extrasbreton{#1}%
    }%
    \@ifundefined{extrasbritish}{}{%
      \addto\extrasbritish{#1}%
    }%
    \@ifundefined{extrasbulgarian}{}{%
      \addto\extrasbulgarian{#1}%
    }%
    \@ifundefined{extrascatalan}{}{%
      \addto\extrascatalan{#1}%
    }%
    \@ifundefined{extrascroatian}{}{%
      \addto\extrascroatian{#1}%
    }%
    \@ifundefined{extrasczech}{}{%
      \addto\extrasczech{#1}%
    }%
    \@ifundefined{extrasdanish}{}{%
      \addto\extrasdanish{#1}%
    }%
    \@ifundefined{extrasesperanto}{}{%
      \addto\extrasesperanto{#1}%
    }%
    \@ifundefined{extrasestonian}{}{%
      \addto\extrasestonian{#1}%
    }%
    \@ifundefined{extrasfinnish}{}{%
      \addto\extrasfinnish{#1}%
    }%
    \@ifundefined{extrasfrench}{}{%
      \addto\extrasfrench{#1}%
    }%
    \@ifundefined{extrasgalician}{}{%
      \addto\extrasgalician{#1}%
    }%
    \@ifundefined{extrasgerman}{}{%
      \addto\extrasgerman{#1}%
    }%
    \@ifundefined{extrasgreek}{}{%
      \addto\extrasgreek{#1}%
    }%
    \@ifundefined{extrashebrew}{}{%
      \addto\extrashebrew{#1}%
    }%
    \@ifundefined{extrasicelandic}{}{%
      \addto\extrasicelandic{#1}%
    }%
    \@ifundefined{extrasirish}{}{%
      \addto\extrasirish{#1}%
    }%
    \@ifundefined{extrasitalian}{}{%
      \addto\extrasitalian{#1}%
    }%
    \@ifundefined{extraslatin}{}{%
      \addto\extraslatin{#1}%
    }%
    \@ifundefined{extraslsorbian}{}{%
      \addto\extraslsorbian{#1}%
    }%
    \@ifundefined{extrasmagyar}{}{%
      \addto\extrasmagyar{#1}%
    }%
    \@ifundefined{extrasnaustrian}{}{%
      \addto\extrasnaustrian{#1}%
    }%
    \@ifundefined{extrasngerman}{}{%
      \addto\extrasngerman{#1}%
    }%
    \@ifundefined{extrasnorsk}{}{%
      \addto\extrasnorsk{#1}%
    }%
    \@ifundefined{extraspolish}{}{%
      \addto\extraspolish{#1}%
    }%
    \@ifundefined{extrasportuges}{}{%
      \addto\extrasportuges{#1}%
    }%
    \@ifundefined{extrasromanian}{}{%
      \addto\extrasromanian{#1}%
    }%
    \@ifundefined{extrasrussian}{}{%
      \addto\extrasrussian{#1}%
    }%
    \@ifundefined{extrassamin}{}{%
      \addto\extrassamin{#1}%
    }%
    \@ifundefined{extrasscottish}{}{%
      \addto\extrasscottish{#1}%
    }%
    \@ifundefined{extrasserbian}{}{%
      \addto\extrasserbian{#1}%
    }%
    \@ifundefined{extrasslovak}{}{%
      \addto\extrasslovak{#1}%
    }%
    \@ifundefined{extrasslovene}{}{%
      \addto\extrasslovene{#1}%
    }%
    \@ifundefined{extrasspanish}{}{%
      \addto\extrasspanish{#1}%
    }%
    \@ifundefined{extrasswedish}{}{%
      \addto\extrasswedish{#1}%
    }%
    \@ifundefined{extrasturkish}{}{%
      \addto\extrasturkish{#1}%
    }%
    \@ifundefined{extrasUKenglish}{}{%
      \addto\extrasUKenglish{#1}%
    }%
    \@ifundefined{extrasukraine}{}{%
      \addto\extrasukraine{#1}%
    }%
    \@ifundefined{extrasUSenglish}{}{%
      \addto\extrasUSenglish{#1}%
    }%
    \@ifundefined{extrasusorbian}{}{%
      \addto\extrasusorbian{#1}%
    }%
    \@ifundefined{extraswelsh}{}{%
      \addto\extraswelsh{#1}%
    }%
  }
}{%
  \newcommand*{\dt@addtoextras}[1]{}
}
\newcommand*{\setdefaultdate}[1]{\dt@addtoextras{#1}#1}
\RequirePackage{fmtcount}
\DeclareOption{long}{\setdefaultdate{\longdate}}
\DeclareOption{short}{\setdefaultdate{\shortdate}}
\DeclareOption{yyyymmdd}{\setdefaultdate{\yyyymmdddate}}
\DeclareOption{ddmmyyyy}{\setdefaultdate{\ddmmyyyydate}}
\DeclareOption{dmyyyy}{\setdefaultdate{\dmyyyydate}}
\DeclareOption{ddmmyy}{\setdefaultdate{\ddmmyydate}}
\DeclareOption{dmyy}{\setdefaultdate{\dmyydate}}
\DeclareOption{text}{\setdefaultdate{\textdate}}
\DeclareOption{us}{\setdefaultdate{\usdate}}
\DeclareOption{mmddyyyy}{\setdefaultdate{\mmddyyyydate}}
\DeclareOption{mdyyyy}{\setdefaultdate{\mdyyyydate}}
\DeclareOption{mmddyy}{\setdefaultdate{\mmddyydate}}
\DeclareOption{mdyy}{\setdefaultdate{\mdyydate}}
\DeclareOption{level}{\fmtcountsetoptions{fmtord=level}}
\DeclareOption{raise}{\fmtcountsetoptions{fmtord=raise}}
\DeclareOption{dayofweek}{\showdowtrue}
\DeclareOption{nodayofweek}{\showdowfalse}
\DeclareOption{nodate}{\@dt@nodatetrue}
\DeclareOption{hhmmss}{\settimeformat{hhmmsstime}}
\DeclareOption{24hr}{\settimeformat{xxivtime}}
\DeclareOption{12hr}{\settimeformat{ampmtime}}
\DeclareOption{oclock}{\settimeformat{oclock}}
\DeclareOption{iso}{%
  \setdefaultdate{\yyyymmdddate}\settimeformat{hhmmsstime}%
  \renewcommand*{\dateseparator}{-}%
  \renewcommand*{\timeseparator}{:}%
}
\newcommand*{\loadDTdef}[1]{%
  \@ifundefined{ver@dt-#1.def}%
  {%
    \InputIfFileExists{dt-#1.def}%
    {}%
    {%
       \PackageWarning{datetime}{Can't find datetime language
         definition file for `#1'}%
    }%
  }%
  {}%
}
\DeclareOption{austrian}{\loadDTdef{austrian}}
\DeclareOption{american}{\loadDTdef{american}}
\DeclareOption{bahasa}{\loadDTdef{bahasa}}
\DeclareOption{basque}{\loadDTdef{basque}}
\DeclareOption{breton}{\loadDTdef{breton}}
\DeclareOption{british}{\loadDTdef{british}}
\DeclareOption{bulgarian}{\loadDTdef{bulgarian}}
\DeclareOption{catalan}{\loadDTdef{catalan}}
\DeclareOption{croatian}{\loadDTdef{croatian}}
\DeclareOption{czech}{\loadDTdef{czech}}
\DeclareOption{danish}{\loadDTdef{danish}}
\DeclareOption{dutch}{\loadDTdef{dutch}}
\DeclareOption{esperanto}{\loadDTdef{esperanto}}
\DeclareOption{estonian}{\loadDTdef{estonian}}
\DeclareOption{finnish}{\loadDTdef{finnish}}
\DeclareOption{french}{\loadDTdef{french}}
\DeclareOption{galician}{\loadDTdef{galician}}
\DeclareOption{german}{\loadDTdef{german}\dategerman}
\DeclareOption{greek}{\loadDTdef{greek}}
\DeclareOption{hebrew}{\loadDTdef{hebrew}}
\DeclareOption{icelandic}{\loadDTdef{icelandic}}
\DeclareOption{irish}{\loadDTdef{irish}}
\DeclareOption{italian}{\loadDTdef{italian}}
\DeclareOption{latin}{\loadDTdef{latin}}
\DeclareOption{lsorbian}{\loadDTdef{lsorbian}}
\DeclareOption{magyar}{\loadDTdef{magyar}}
\DeclareOption{naustrian}{\loadDTdef{naustrian}}
\DeclareOption{ngerman}{\loadDTdef{ngerman}\datengerman}
\DeclareOption{norsk}{\loadDTdef{norsk}}
\DeclareOption{polish}{\loadDTdef{polish}}
\DeclareOption{portuges}{\loadDTdef{portuges}}
\DeclareOption{romanian}{\loadDTdef{romanian}}
\DeclareOption{russian}{\loadDTdef{russian}}
\DeclareOption{samin}{\loadDTdef{samin}}
\DeclareOption{scottish}{\loadDTdef{scottish}}
\DeclareOption{serbian}{\loadDTdef{serbian}}
\DeclareOption{slovak}{\loadDTdef{slovak}}
\DeclareOption{slovene}{\loadDTdef{slovene}}
\DeclareOption{spanish}{\loadDTdef{spanish}}
\DeclareOption{swedish}{\loadDTdef{swedish}}
\DeclareOption{turkish}{\loadDTdef{turkish}}
\DeclareOption{ukraineb}{\loadDTdef{ukraineb}}
\DeclareOption{usorbian}{\loadDTdef{usorbian}}
\DeclareOption{UKenglish}{\loadDTdef{UKenglish}}
\DeclareOption{USenglish}{\loadDTdef{USenglish}}
\DeclareOption{welsh}{\loadDTdef{welsh}}
\ProcessOptions
\if@dt@nodate
  \PackageInfo{datetime}{option "nodate" used, so note defining
\string\dateformat}
\else
\providecommand*{\dateformat}[4]{%
\@day=#2\relax\@month=#3\relax\@year=#4\relax#1}
\fi
\if@dt@nodate
\PackageInfo{datetime}{option "nodate" used, so not defining
\string\monthname}
\else
\providecommand*{\monthnameenglish}[1][\month]{%
\@orgargctr=#1\relax
\ifcase\@orgargctr
\PackageError{datetime}{Invalid Month number \the\@orgargctr}{Month
numbers should go from 1 (January) to 12 (December)}%
\or January%
\or February%
\or March%
\or April%
\or May%
\or June%
\or July%
\or August%
\or September%
\or October%
\or November%
\or December%
\else \PackageError{datetime}{Invalid Month number \the\@orgargctr}{%
Month numbers should go from 1 (January) to 12 (December)}%
\fi}
\newcommand*{\monthname}[1][\month]{%
\if@dt@multilingual
\@ifundefined{monthname\languagename}{%
\PackageWarning{datetime}{No month names provided for language
'\languagename'}%
\monthnameenglish[#1]}{\csname monthname\languagename\endcsname[#1]}%
\else
\monthnameenglish[#1]%
\fi}
\fi
\if@dt@nodate
\PackageInfo{datetime}{option "nodate" used, so not defining
\protect\shortmonthname}
\else
\providecommand*{\shortmonthnameenglish}[1][\month]{%
\@orgargctr=#1\relax
\ifcase\@orgargctr
\PackageError{datetime}{Invalid Month number \the\@orgargctr}{Month
numbers should go from 1 (jan) to 12 (dec)}%
\or Jan%
\or Feb%
\or Mar%
\or Apr%
\or May%
\or Jun%
\or Jul%
\or Aug%
\or Sept%
\or Oct%
\or Nov%
\or Dec%
\else%
\PackageError{datetime}{Invalid Month number \the\@orgargctr}{Month
numbers should go from 1 (jan) to 12 (dec)}%
\fi}
\newcommand*{\shortmonthname}[1][\month]{%
\if@dt@multilingual
\@ifundefined{shortmonthname\languagename}{%
\PackageWarning{datetime}{No abbreviated month name defined for
language '\languagename', using full version instead}%
\monthname[#1]}{%
\csname shortmonthname\languagename\endcsname[#1]}%
\else
\shortmonthnameenglish[#1]%
\fi}
\fi
\newif\ifleapyear
\newcount\@dtctr
\if@dt@nodate
\PackageInfo{datetime}{option "nodate" used, so not defining
\string\dayofweek \space or \string\shortdayofweek}
\else
\providecommand*{\testifleapyear}[1]{%
\leapyearfalse
\@year=#1\relax
\@dtctr=\@year
\@modulo{\@dtctr}{400}%
\ifnum\@dtctr=0\relax
\leapyeartrue %         year mod 400 = 0 => leap year
\else
\@dtctr=\@year
\@modulo{\@dtctr}{100}%
\ifnum\@dtctr=0\relax
\leapyearfalse %        year mod 100 = 0 && year mod 400 != 0 => not a leap year
\else
\@dtctr=\@year
\@modulo{\@dtctr}{4}%
\ifnum\@dtctr=0\relax
\leapyeartrue %         year mod 4 = 0 && year mod 100 != 0 => leap year
\fi
\fi
\fi
}
\newcount\dayofyear
\providecommand*{\computedayofyear}[3]{%
\testifleapyear{#3}%
\dayofyear=0\relax
\@day=#1\relax \@month=#2\relax \@year=#3\relax
\ifcase\@month
\or
\or \advance\dayofyear by 31\relax
\or \advance\dayofyear by 59\relax
\or \advance\dayofyear by 90\relax
\or \advance\dayofyear by 120\relax
\or \advance\dayofyear by 151\relax
\or \advance\dayofyear by 181\relax
\or \advance\dayofyear by 212\relax
\or \advance\dayofyear by 243\relax
\or \advance\dayofyear by 273\relax
\or \advance\dayofyear by 304\relax
\or \advance\dayofyear by 334\relax
\else
\PackageError{datetime}{Invalid month number}{The second argument to
\string\computedayofyear \space should lie in the range 1-12}%
\fi
\ifnum\@month>2\relax
\ifleapyear\advance\dayofyear by 1\relax\fi
\fi
\advance\dayofyear by \@day\relax
}
\newcount\dayofweek
\providecommand*{\computedayofweek}[3]{%
\computedayofyear{#1}{#2}{#3}%
\@dtctr=#3\relax
\advance\@dtctr by -1901\relax
\@modulo{\@dtctr}{28}%
\dayofweek=\@dtctr
\divide\dayofweek by 4\relax
\advance\dayofweek by \@dtctr
\advance\dayofweek by 2\relax
\@modulo{\dayofweek}{7}%
\advance\dayofweek by \dayofyear
\advance\dayofweek by -1\relax
\@modulo{\dayofweek}{7}%
\advance\dayofweek by 1\relax}
\providecommand*{\dayofweeknameidenglish}[1]{%
\ifcase#1\relax
\or Sunday%
\or Monday%
\or Tuesday%
\or Wednesday%
\or Thursday%
\or Friday%
\or Saturday%
\fi}
\providecommand*{\dayofweeknameid}[1]{%
\if@dt@multilingual
\@ifundefined{dayofweeknameid\languagename}{%
\ifthenelse{\equal{\languagename}{nohyphenation}}{}{%
\PackageWarning{datetime}{No week day names defined for language
'\languagename', defaulting to English}}%
\dayofweeknameidenglish{#1}}{%
\csname dayofweeknameid\languagename\endcsname{#1}}%
\else
\dayofweeknameidenglish{#1}%
\fi
}
\providecommand*{\dayofweekname}[3]{%
\computedayofweek{#1}{#2}{#3}%
\dayofweeknameid{\dayofweek}%
}
\providecommand*{\thisdayofweekname}{%
\dayofweekname{\day}{\month}{\year}}
\providecommand*{\shortdayofweeknameidenglish}[1]{%
\ifcase#1\relax
\or Sun%
\or Mon%
\or Tue%
\or Wed%
\or Thu%
\or Fri%
\or Sat%
\fi}
\providecommand*{\shortdayofweekname}[3]{%
\computedayofweek{#1}{#2}{#3}%
\if@dt@multilingual
\@ifundefined{shortdayofweeknameid\languagename}{%
\ifthenelse{\equal{\languagename}{nohyphenation}}{}{%
\PackageWarning{datetime}{No abbreviated week day names defined for
language '\languagename', defaulting to long version}}%
\dayofweeknameid{\dayofweek}}{%
\csname shortdayofweeknameid\languagename\endcsname\dayofweek}%
\else
\shortdayofweeknameidenglish{\dayofweek}%
\fi
}
\providecommand*{\thisshortdayofweekname}{%
\dayofweekname{\day}{\month}{\year}}
\fi
\if@dt@nodate
\else
\DeclareRobustCommand*{\today}{\formatdate{\day}{\month}{\year}}
\fi
\if@dt@nodate
\else
\@ifundefined{dateenglish}{}{\let\dateenglish\longdate}
\@ifundefined{dateUKenglish}{}{\loadDTdef{UKenglish}}
\@ifundefined{dateUSenglish}{}{\loadDTdef{USenglish}}
\@ifundefined{datebritish}{}{\loadDTdef{british}}
\@ifundefined{dateamerican}{}{\loadDTdef{american}}
\@ifundefined{dateaustrian}{}{\loadDTdef{austrian}}
\@ifundefined{datebahasa}{}{\loadDTdef{bahasa}}
\@ifundefined{datebasque}{}{\loadDTdef{basque}}
\@ifundefined{datebreton}{}{\loadDTdef{breton}}
\@ifundefined{datebulgarian}{}{\loadDTdef{bulgarian}}
\@ifundefined{datecatalan}{}{\loadDTdef{catalan}}
\@ifundefined{datecroatian}{}{\loadDTdef{croatian}}
\@ifundefined{dateczech}{}{\loadDTdef{czech}}
\@ifundefined{datedanish}{}{\loadDTdef{danish}}
\@ifundefined{datedutch}{}{\loadDTdef{dutch}}
\@ifundefined{dateesperanto}{}{\loadDTdef{esperanto}}
\@ifundefined{dateestonian}{}{\loadDTdef{estonian}}
\@ifundefined{datefinnish}{}{\loadDTdef{finnish}}
\@ifundefined{datefrench}{}{\loadDTdef{french}}
\@ifundefined{dategalician}{}{\loadDTdef{galician}}
\@ifundefined{dategerman}{}{\loadDTdef{german}}
\@ifundefined{dategreek}{}{\loadDTdef{greek}}
\@ifundefined{datehebrew}{}{\loadDTdef{hebrew}}
\@ifundefined{dateicelandic}{}{\loadDTdef{icelandic}}
\@ifundefined{dateirish}{}{\loadDTdef{irish}}
\@ifundefined{dateitalian}{}{\loadDTdef{italian}}
\@ifundefined{datelatin}{}{\loadDTdef{latin}}
\@ifundefined{datelsorbian}{}{\loadDTdef{lsorbian}}
\@ifundefined{datemagyar}{}{\loadDTdef{magyar}}
\@ifundefined{datenaustrian}{}{\loadDTdef{naustrian}}
\@ifundefined{datengerman}{}{\loadDTdef{ngerman}}
\@ifundefined{datenorsk}{}{\loadDTdef{norsk}}
\@ifundefined{datepolish}{}{\loadDTdef{polish}}
\@ifundefined{dateportuges}{}{\loadDTdef{portuges}}
\@ifundefined{dateromanian}{}{\loadDTdef{romanian}}
\@ifundefined{daterussian}{}{\loadDTdef{russian}}
\@ifundefined{datesamin}{}{\loadDTdef{samin}}
\@ifundefined{datescottish}{}{\loadDTdef{scottish}}
\@ifundefined{dateserbian}{}{\loadDTdef{serbian}}
\@ifundefined{dateslovak}{}{\loadDTdef{slovak}}
\@ifundefined{dateslovene}{}{\loadDTdef{slovene}}
\@ifundefined{datespanish}{}{\loadDTdef{spanish}}
\@ifundefined{dateswedish}{}{\loadDTdef{swedish}}
\@ifundefined{dateturkish}{}{\loadDTdef{turkish}}
\@ifundefined{dateukraineb}{}{\loadDTdef{ukraineb}}
\@ifundefined{dateusorbian}{}{\loadDTdef{usorbian}}
\@ifundefined{datewelsh}{}{\loadDTdef{welsh}}
\fi
\newtoks\dt@a \newtoks\dt@b
\edef\pdfdate{\the\year}
\dt@b=\expandafter{\pdfdate}
\dt@a=\expandafter{\the\month}
\ifnum\month<10\relax
\edef\pdfdate{\the\dt@b0\the\dt@a}
\else
\edef\pdfdate{\the\dt@b\the\dt@a}
\fi
\dt@b=\expandafter{\pdfdate}
\dt@a=\expandafter{\the\day}
\ifnum\day<10\relax
\edef\pdfdate{\the\dt@b0\the\dt@a}
\else
\edef\pdfdate{\the\dt@b\the\dt@a}
\fi
\@dtctr=\time%
\divide\@dtctr by 60\relax
\dt@b=\expandafter{\pdfdate}
\dt@a=\expandafter{\the\@dtctr}
\ifnum\@dtctr<10
\edef\pdfdate{\the\dt@b0\the\dt@a}
\else
\edef\pdfdate{\the\dt@b\the\dt@a}
\fi
\@dtctr=\time%
\@modulo{\@dtctr}{60}%
\dt@b=\expandafter{\pdfdate}
\dt@a=\expandafter{\the\@dtctr}
\ifnum\@dtctr<10\relax
\edef\pdfdate{\the\dt@b0\the\dt@a}
\else
\edef\pdfdate{\the\dt@b\the\dt@a}
\fi
\dt@a={00}
\dt@b=\expandafter{\pdfdate}
\edef\pdfdate{\the\dt@b\the\dt@a}
\endinput
%%
%% End of file `datetime.sty'.
}
\end{verbatim}
This, I agree, is an unpleasant cludge.

\item The commands \cs{monthname}\meta{language} are not 
implemented.

\item Some of the languages are not implemented.

\item The package option \pkgopt{nodate} is not implemented.

\end{itemize}

\section{Troubleshooting}

There is a \sty{datetime} FAQ available at:
\url{http://theoval.cmp.uea.ac.uk/~nlct/latex/packages/faq/}

\StopEventually{\phantomsection
\addcontentsline{toc}{section}{Index}\PrintIndex
}

\end{document}
