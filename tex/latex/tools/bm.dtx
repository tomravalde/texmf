% \iffalse meta-comment
%
% Copyright 1993 1994 1995 1996 1997 1998 1999 2000 2001 2002 2003 2004 2005
% 2006 2008 2009
% The LaTeX3 Project and any individual authors listed elsewhere
% in this file. 
% 
% This file is part of the Standard LaTeX `Tools Bundle'.
% -------------------------------------------------------
% 
% It may be distributed and/or modified under the
% conditions of the LaTeX Project Public License, either version 1.3c
% of this license or (at your option) any later version.
% The latest version of this license is in
%    http://www.latex-project.org/lppl.txt
% and version 1.3c or later is part of all distributions of LaTeX 
% version 2005/12/01 or later.
% 
% The list of all files belonging to the LaTeX `Tools Bundle' is
% given in the file `manifest.txt'.
% 
% \fi
%
% \iffalse
%% Copyright 1996 1997 1998 1999 2002 2003 2004 David Carlisle Frank Mittelbach
%%
%% Development of this package was commissioned by Y&Y Inc.
%% http://www.yandy.com
%
%
%<*dtx>
          \ProvidesFile{bm.dtx}
%</dtx>
%<package>\NeedsTeXFormat{LaTeX2e}
%<package>\ProvidesPackage{bm}
%<driver>\ProvidesFile{bm.drv}
% \fi
%         \ProvidesFile{bm.dtx}
          [2004/02/26 v1.1c Bold Symbol Support (DPC/FMi)]
%
% \iffalse
%<*driver>
\documentclass{ltxdoc}
\usepackage{bm}
\begin{document}
\DocInput{bm.dtx}
\end{document}
%</driver>
% \fi
%
% \CheckSum{944}
%
% \GetFileInfo{bm.dtx}
%
% \title{The \textsf{bm} package\thanks{This file
%         has version number \fileversion, last
%         revised \filedate.}\ \thanks{Development of this package
%         was commissioned by Y\&Y.}}
%
% \date{\filedate}
% \author{David Carlisle with support by Frank Mittelbach}
%  \maketitle
%
% \section{Introduction}
%
% This package defines commands to access bold math
% symbols. The basic command is |\bm| which may be used to make the
% math expression in its argument be typeset using bold fonts.
%
% The syntax of |\bm| is:\\
% |\bm|\marg{math expression}\\
% So |$\alpha \not= \bm{\alpha}$| produces $\alpha \not= \bm{\alpha}$.
%
% |\bm| goes to some trouble to preserve the spacing, so that for
% instance |\bm<| is a bold $\bm<$ but with the correct |\mathrel|
% spacing that \TeX\ gives to $<$. The calculations that \TeX\ needs to
% do for |\bm| can be quite involved and so a definition form is
% provided.
%
% |\DeclareBoldMathCommand|%^^A
%       \oarg{math version}\marg{cmd}\marg{math expression}
%
% Defines |\cmd| to be the bold form of the math expression.
% The \meta{math version} defaults to `bold' (i.e., |\boldmath|).
%
% For relatively simple expressions, the resulting definitions are very
% efficient, for instance after:\\
% |\DeclareBoldMathCommand\balpha{\alpha}|\\
% |\balpha| is a single `mathchardef' token producing a bold alpha,
% and so is just as fast to execute as |\alpha|.
%
% The above command is mainly intended for use in packages.
% For occasional use in \LaTeX\ documents, and for compatibility
% with the plain \TeX\ support for the mathtime fonts, a `user-level'
% version, |\bmdefine| is provided that is equivalent to:\\
% |\DeclareBoldMathCommand[bold]|.
%
% If there is a `heavy' math version defined (usually accessed by a
% user-command |\heavymath|) then a similar command |\hm|
% is defined which access these `ultra bold' fonts. Currently this is
% probably only useful with the `mathtime plus' font collection.
% Definitions of commands that use these fonts may be made by
% specifying the optional argument `heavy' to |\DeclareBoldMathCommand|.
% Again an abbreviation, |\hmdefine|, is provided, equivalent to:\\
% |\DeclareBoldMathCommand[heavy]|.
%
% The command names (but not the implementation) are taken from Michael
% Spivak's macros to support the mathtime fonts for plain \TeX. In those
% original macros, the syntax for |\bmdefine| was
% |\bmdefine\balpha{\bm\alpha}| (with a nested |\bm|). This syntax also
% works with this package.
%
% \section{Font allocation}
% In order to access bold fonts in the simplest and quickest possible
% manner, the package normally allocates symbol fonts for bold
% (and possibly heavy) fonts into the `normal' math version.
% By default it allocates at most four fonts for |\bm| and at most
% three fonts for |\hm|. This means that if the mathtime plus font set
% is being used, seven additional symbol fonts will be used, in addition
% to the basic four that \LaTeX\ already declares. The mathtime
% package also declares an extra symbol font, bringing the total to
% twelve. The maximum number of symbol \emph{and} math alphabet fonts
% that can be used in a math version is sixteen. So the above allocation
% scheme does not leave room for many extra math symbols (such as the
% AMS symbols) or math alphabets (such as |\mathit|).
%
% Before loading the \textsf{bm} package you may define |\bmmax|
% and |\hmmax| to be suitable values, for instance you may want
% to set |\newcommand\hmmax{0}| if you will not be using |\hm|
% much, but you do have a heavy math version defined.
%
% Even if |\bmmax| is set to zero, |\bm| will still access the correct
% bold fonts (by accessing the fonts via |\boldmath|) but this method
% is slower, and does not work with delimiters. Delimiters can only be
% made bold if the bold font has been allocated.
%
% Conversely if you have a non standard font set that makes available
% extra math delimiters and accents in bold and medium weights you may
% want to \emph{increase} |\bmmax| so that fonts are allocated for
% your font set.
%
% \section{Features}
% In most cases this package should work in a fairly self explanatory
% way, but there are some things that might not be obvious.
%
% \subsection{Interaction with Math Alphabet Commands}
%
% As mentioned above, |\bm| goes to some trouble to try to make a
% command that is just like its argument, but using a bold font.
% This does not always produce the effect that you might expect.
%\begin{verbatim}
% $1 g \bm{g}$
% $2 \mathrm{g \bm{g}}$
% $3 {g} \bm{{g}}$
% $4 \mathrm{{g} \bm{{g}}}$
% $5 \mathrm{g} \bm{\mathrm{g}}$
%\end{verbatim}
% produces the following:
% \begin{flushleft}
% $1 g \bm{g}$
% $2 \mathrm{g \bm{g}}$
% $3 {g} \bm{{g}}$
% $4 \mathrm{{g} \bm{{g}}}$
% $5 \mathrm{g} \bm{\mathrm{g}}$
% \end{flushleft}
% In math mode `g' is effectively a command that produces the letter `g'
% from the `letters' alphabet, unless a Math Alphabet command is in
% effect, in which case the `g' comes from the specified alphabet.
% |\bm{g}| makes an equivalent command, but which defaults to a bold
% letter alphabet. So in the first example |\bm{g}| is bold math italic,
% but in the second example the |\mathrm| applies to both |g| and
% |\bm{g}| in the same way, and so they are both roman.
%
% |\bm| only inspects the `top level' definition of a command, for more
% complicated expressions, and anything inside a |{ }| group, |\bm|
% forces bold fonts by essentially the same (slow) technique used by the
% AMS |\boldsymbol| command (but |\bm| still takes more care of the
% spacing). So the third example produces identical output to the first
% (but \TeX\ takes more time producing it).
%
% In the fourth example the |\mathrm{\bm{g}}| is essentially
% equivalent to |\mathrm{\mbox{\boldmath$g$}}|. Currently math alphabet
% settings are not passed down to `nested' math lists, and so in this
% example, the |\mathrm| has no effect, and a bold math italic $\bm g$
% is obtained.
%
% Similarly the last example is equivalent to
% |$\mbox{\boldmath$\mathrm{g}$}}| and so in this case, one obtains a
% bold roman \textbf{g}.
%
% \subsection{Delimiters}
% \TeX\ can treat character tokens in two\footnote
%  {Well more than two really.}
% ways. If there is a preceding
% |\left| or |\right| it can treat them as a delimiter, otherwise it can
% treat them as a standard character. For example |\left<\right>|
% produces $\left<\right>$, which is totally different from |<>|, which
% produces $<>$.
%
% \TeX\ can only do this for character tokens. Commands such as
% |\langle| do not act in this way. This means that |\bm| has to decide
% whether to treat a character as a delimiter or not. The rule it uses
% is, it makes a delimiter command for a character if the previous
% token in the argument was |\left| or |\right|. So |\left\bm{<}| does
% not work, but |\bm{\left<}| does.
%
% \subsection{Command Arguments}
%
% Normally if a command takes arguments the full command, including
% any arguments, should be included in |\bm|.
%
% So |\bm{\overbrace{abc}}| (producing \smash{$\bm{\overbrace{abc}}$})
% not
% |\bm{\overbrace}{abc}|. If you do not include all the arguments you
% will typically get the error message:\\
% |Runaway argument?|\\
% |! Forbidden control sequence found while scanning use of|
%                                                      \texttt{\ldots}
%
% However commands defined in terms of the \TeX\ accent and
% radical primitives \emph{may} be used without their arguments.
% So |\bm{\hat}{a}| produces $\bm{\hat}{a}$, a bold accent over a
% non-bold $a$ (compare $\hat{a}$)
%  whereas |\bm{\hat{a}}| makes both the $a$ and the accent bold,
% $\bm{\hat{a}}$.
% Similarly, although the \LaTeX\ command |\sqrt| must be used with its 
% arguments, |\sqrtsign| may be used as in |\bm\sqrtsign{abc}| to
% produce $\bm\sqrtsign{abc}$ rather than  $\sqrtsign{abc}$ or
%  $\bm{\sqrtsign{abc}}$
%
% If you really need to make a command with arguments use bold fonts
% without making all of the arguments bold, you can explicitly
% reset the math version in the argument, eg:\\
% $\begin{array}{ccc}
% |\sqrt{xyz}|&|\bm{\sqrt{xyz}}|&|\bm{\sqrt{\mbox{\unboldmath$xyz$}}}|\\
% \sqrt{xyz}& \bm{\sqrt{xyz}}&  \bm{\sqrt{\mbox{\unboldmath$xyz$}}}
% \end{array}$
%
% \subsection{Bold fonts}
% This package interrogates the font allocations of the bold and heavy
% math versions, to determine which bold fonts are available.
% This means that it is best to load the package \emph{after}
% any packages that define new symbol fonts, or (like the
% \textsf{mathtime} package) completely change the symbol font
% allocations.
%
% If no bold font appears to be available for a particular symbol,
% |\bm| will use `poor man's bold' that is, overprinting the same
% character in slightly offset
% positions to give an appearance of boldness.
%
% In the standard Computer Modern font set, there is no bold
% `large symbols' font. In the `mathptm' and (standard) mathtime
% font sets there are no bold math fonts. In the `mathtime plus'
% font set there are suitable fonts for bold and heavy math setting,
% and so |\bm| and |\hm| work well. Similarly in the basic Lucida
% New Math font set there are no bold math fonts, so |\bm| will
% use `poor man's bold. However if the Lucida Expert set is used,
% Then |\bm| will detect, and use the bold math fonts that are
% available.
%
% As discussed above, one may set |\bmmax| higher or lower than its
% default value of four to control the font allocation system. Finer
% control may be gained by explicitly declaring bold symbol fonts.
% Suppose you have a symbol font `xyz' that is available in medium and
% bold weights, then you would declare this to \LaTeX\ via:\\
% |\DeclareSymbolFont{extras}  {OMS}{xyz}{m}{n}|\\
% |\SetSymbolFont{extras}{bold}{OMS}{xyz}{bx}{n}|\\
% At this point the symbols wil be available in the normal math
% version, and their bold variants in |\boldmath|. If you also
% declare:\\ 
% |\DeclareSymbolFont{boldextras}{OMS}{xyz}{bx}{n}|\\
% That is, declare a symbol font whose name is formed by prefixing
% `bold' (or `heavy') to an existing symbol font, then |\bm| (or
% |\hm|) will use this font directly, rather then accessing the
% `extras' symbol font via |\boldmath|.
%
% \subsection{Strange failures}
% In order to get the correct spacing, |\bm| has to `investigate' the
% definition of the commands in its argument. It is possible that
% some strange constructions could `confuse' this investigation.
% If this happens then \LaTeX\ will almost certainly stop with a strange
% error. This should not happen with any of the math symbols
% defined in the base \LaTeX\ or AMS distributions, or any commands
% defined in terms of those symbols using normal \LaTeX\ math
% constructs. However if some command does fail to work inside |\bm|
% you should always be able to surround it with an extra set of braces
% |\bm{{\cmd}}| rather than |\bm{\cmd}|. |\bm| will not then attempt
% to set the correct spacing, so you may need to set it explicitly,
% for instance, for a relation, |\bm{\mathrel{\cmd}}|.
%
% \subsection{AMS package \textsf{amsbsy}}
% The |\bm| command shares some functionailty with the |\boldsymbol|
% command from the AMS \LaTeX\ collection. To aid in moving documents
% between these two packages, this package defines |\boldsymbol| and
% |\heavysymbol| as alternative names for |\bm| and |\hm|.
%
% \StopEventually{}
%
% \section{Implementation}
%
% \changes{v0.01}{1996/12/01}
%      {Initial DPC attempt,
%       remove all assumptions about mathtime encoding}
% \changes{v0.02}{1996/12/02}
%      {Add \cs{mathchoice} stuff}
% \changes{v0.03}{1996/12/12}
%      {Completely reimplement (again). Add \cs{bmdefine}.}
% \changes{v0.04}{1996/12/12}
%      {Add \cs{hm} and support for special active mathcode.}
% \changes{v0.05}{1996/12/12}
%      {Assorted fixes}
% \changes{v0.99}{1997/01/16}
%      {First public version}
% \changes{v1.0a}{1997/02/14}
%      {First ctan version (same as 0.99b)}
% \changes{v1.0b}{1997/04/14}
%      {Add to tools bundle.}
% \changes{v1.0g}{1999/07/05}
%      {minor doc changes latex/3058}
%
%
%
% The commands |\bm| and |\hm| work by defining a number of additional
% symbol fonts corresponding to the standard ones
%  `operators', `letters', `symbols', and `largesymbols'.
% The names for these symbols fonts are produced by prefixing the usual
% name with `bold' or `heavy'.
%
% For maximum flexibility we get the font definitions by looking in the
% corresponding math versions, i.e., into |\mv@bold| and if defined into
% |\mv@heavy|.
%
%    \begin{macrocode}
%<*package>
%    \end{macrocode}
% \changes{v0.09}{1996/12/19}
%      {Always define \cs{bm} even if no bold math}
%
% \begin{macro}{\bm@table}
% \changes{v0.10}{1997/01/04}
%      {Macro added}
% \changes{v0.12}{1997/01/10}
%      {Ensure do not allocate too many math group slots.}
% \changes{v0.99a}{1997/01/17}
%      {Let \cs{hm} use font allocated for \cs{bm}}
% \changes{v0.99b}{1997/01/19}
%      {Restore NFSS internals}
% \begin{macro}{\bm@boldtable}
% \begin{macro}{\bm@heavytable}
% The table, |\bm@table|, (which is locally |\let| to either the bold
% or heavy version) defines, for each \meta{math group} (\meta{fam}),
% the `offset' to the bold version of the specified symbol font.
% If there is no bold symbol font defined, the offset will be
% set to zero if there is a bold font assigned to this slot in the
% bold math version, or $-1$ if the font in the bold math version
% is the same as the one in the normal math version. In this case
% a `poor man's bold' system of overprinting is used to achieve
% boldness where this is possible.
%
% The settings are made at the time this package is read, and so
% it is best to load this package late, after any font loading packages
% have been loaded. Symbol fonts loaded after this package will get the
% offset of zero, so they will still be made bold by |\bm| as long as an
% appropriate font is declared for the bold math version.
%
% |\bm@boldtable| and |\bm@heavytable| are set up using very similar
% code, which is temporarily defined to |\bm|, to save wasting a csname.
% Similarly |\bm@pmb|\ldots\ (which will be defined later) are used
% as scratch macros.
%
% The general plan. Run through the fonts allocated to the normal math
% version. Ignore \meta{math alphabet} allocations\footnote{For now?}
% but for each math symbol font, look in the math version specified by
% |#1| (bold or heavy). If the font there is different, then allocate
% a new symbol font in the normal math version to access that bold font
% and place the numerical difference between the allocations of the bold
% and normal font into the table being built (|\bm@boldtable|, if |#1|
% is bold). If the symbol allocation is already greater than |\bmmax|
% do not allocate a new symbol font, but rather set the offset in the
% table to zero. |\bm| will detect this, and use |\boldmath| on its
% argument in this case, so the bold font will be accessed but more
% slowly than using a direct access to a bold font allocated into the
% normal math version. If the font allocated in the bold math version is
% the same as the font in the normal math version, set the offset to
% $-1$, which is a flag value that causes |\bm| to use `poor man's bold'
% overprinting three copies of the symbol, offset slightly to give an
% appearance of boldness.
%
% Fonts containing delimiters and math accents \emph{must} be allocated
% into the normal math version if they are to be used with |\bm|. (In
% these cases |\bm| will produce the normal weight symbol, rather than
% using |\boldmath| or poor man's bold.)
%
%    \begin{macrocode}
\def\bm#1#2{%
%    \end{macrocode}
% This code can not work inside a group, as that would affect any symbol
% font allocations, so instead use some scratch macros to save and
% restore the definitions of commands we need to change locally.
%    \begin{macrocode}
  \let\bm@pmb\install@mathalphabet
  \let\bm@pmb@\getanddefine@fonts
  \let\bm@pmb@@\or
  \edef\bm@general{\f@encoding/\f@family/\f@series/\f@shape/\f@size}%
%    \end{macrocode}
%
% |#2| specifies the maximum number of fonts to allocate
% (either |\bmmax| or |\hmmax|). First check against |\count18|
% that there are that many slots left, and if not reduce accordingly.
% Put the resulting value in |\@tempcnta|.
%    \begin{macrocode}
  \@tempcnta#2%
  \count@-\count18%
  \advance\count@-\@tempcnta
  \advance\count@15\relax
  \ifnum\count@<\z@
    \advance\@tempcnta\count@
  \fi
%    \end{macrocode}
% Make |\or| non-expandable, so we can build an |\ifcase| bit-by-bit
% in a sequence of |\edef|s.
%    \begin{macrocode}
  \let\or\relax
%    \end{macrocode}
%
% Initialise the table (to |\@gobble| to remove the first |\or|).
%    \begin{macrocode}
  \expandafter\let\csname bm@#1table\endcsname\@gobble
%    \end{macrocode}
%
% Helper macro that adds the next entry to the table being built.
%    \begin{macrocode}
  \def\bm@define##1{%
    \expandafter\xdef\csname bm@#1table\endcsname{%
      \csname bm@#1table\endcsname\or##1}}%
%    \end{macrocode}
%
% Each symbol font is recorded in the math version list by a sequence
% such as:\\
% | \getanddefine@fonts \symsymbols \OMS/cmsy/m/n|\\
% Where the first argument is a chardef token carrying the number
% allocated (to symbols, in this example), and the second argument is a
% csname whose \emph{name} denotes the font used.
% So locally redefine |\getanddefine@fonts| to compare |#2| with
% the name in the appropriate slot in the bold math version.
%    \begin{macrocode}
  \def\getanddefine@fonts##1##2{%
    \def\@tempa{##2}%
%    \end{macrocode}
%
%    \begin{macrocode}
    \def\@tempb####1##1####2####3\@nil{\def\@tempb{####2}}%
%    \end{macrocode}
%
%    \begin{macrocode}
    \expandafter\expandafter\expandafter
      \@tempb\csname mv@#1\endcsname\@nil
%    \end{macrocode}
%
% Now |\@tempa| and  |\@tempb| contain the names of the fonts allocated
% to this slot in the two math versions.
%    \begin{macrocode}
    \ifx\@tempa\@tempb
%    \end{macrocode}
% If they are the same, set this offset to $-1$, as a flag to use
% poor man's bold.
%    \begin{macrocode}
      \bm@define\m@ne
    \else
%    \end{macrocode}
% Else make a new name by adjoining |#1| to the name of the symbol font
% eg, |\symboldsymbols| to match |\symsymbols|. If that font has already
% been allocated, or if |\@tempcnta| is positive so we can allocate a
% new slot for this font, then the table will be
% set with the offset between the two fonts. otherwise set the offset to
% zero (so |\boldmath| will be used to access the font).
%    \begin{macrocode}
      \edef\@tempa{sym#1\expandafter\@gobblefour\string##1}%
      \ifnum\@tempcnta<%
           \expandafter\ifx\csname\@tempa\endcsname\relax
             \@ne
           \else
             \m@ne
           \fi
%    \end{macrocode}
%
%    \begin{macrocode}
        \bm@define\z@
      \else
%    \end{macrocode}
%
% If the font is not yet allocated, allocate it now, using
% an internal hack into |\DeclareMathSymbolFont|.
%
% However before allocating it look in the bold math version
% to see if it is the same, and if so use that. For example
% with Mathtime the `operators' font in the `heavy' math version
% is different from that in `normal', but it is the same as
% the font in `bold' (Times bold). So rather than allocate
% |\symheavyoperators| just set it equal to |\symboldoperators|.
%    \begin{macrocode}
        \expandafter\ifx\csname\@tempa\endcsname\relax
          \begingroup
          \escapechar\m@ne
          \edef\@tempb{\endgroup
            \noexpand\split@name
            \expandafter\string\@tempb}%
          \@tempb/\@nil
%    \end{macrocode}
%
%    \begin{macrocode}
         \expandafter\ifx
           \csname symbold\expandafter\@gobblefour\string##1\endcsname
           \relax
%    \end{macrocode}
%  If no font has been allocated for |\bm| yet, then allocate it now.
%    \begin{macrocode}
            \expandafter\new@mathgroup\csname\@tempa\endcsname
            \expandafter\new@symbolfont\csname\@tempa\endcsname
                       \f@encoding\f@family\f@series\f@shape
%    \end{macrocode}
% Reduce by one the number of fonts we can still allocate.
%    \begin{macrocode}
            \advance\@tempcnta\m@ne
%    \end{macrocode}
%
%    \begin{macrocode}
         \else
%    \end{macrocode}
% Else do a similar look into the bold mathgroup.
% Use |\bm@expand| as a scratch macro to save on string space.
%    \begin{macrocode}
           \def\bm@expand####1##1####2####3\@nil{\def\bm@expand{####2}}%
    \expandafter\expandafter\expandafter
           \bm@expand\csname mv@bold\endcsname\@nil
%    \end{macrocode}
%
%    \begin{macrocode}
           \ifx\bm@expand\@tempb
%    \end{macrocode}
% If the font just found (in heavy) is the same as the font in bold
% use the slot (in normal) previously allocated for the bold font.
% (That clear?)
%    \begin{macrocode}
             \expandafter\let\csname\@tempa\expandafter\endcsname
                \csname symbold\expandafter
                        \@gobblefour\string##1\endcsname
%    \end{macrocode}
%
%    \begin{macrocode}
           \else
%    \end{macrocode}
% Otherwise allocate a new slot for it.
%    \begin{macrocode}
             \expandafter\new@mathgroup\csname\@tempa\endcsname
             \expandafter\new@symbolfont\csname\@tempa\endcsname
                       \f@encoding\f@family\f@series\f@shape
             \advance\@tempcnta\m@ne
%    \end{macrocode}
%
%    \begin{macrocode}
           \fi
         \fi
%    \end{macrocode}
%
%    \begin{macrocode}
        \else
%    \end{macrocode}
% If the font has been allocated already, use the existing allocation.
%    \begin{macrocode}
          \PackageInfo{bm}%
            {Symbol font \@tempa\space already defined.\MessageBreak
             Not overwriting it}%
        \fi
%    \end{macrocode}
% Whether the font has just been allocated, or whether it was previously
% allocated, compute the offset and add it to the table.
%    \begin{macrocode}
        \count@\csname\@tempa\endcsname
        \advance\count@-##1%
        \bm@define{\the\count@\relax}%
      \fi
    \fi}%
%    \end{macrocode}
%
% The math version list also contains information about math alphabet
% commands, but we want to ignore those  here, so \ldots
%    \begin{macrocode}
  \let\install@mathalphabet\@gobbletwo
%    \end{macrocode}
%
% Having set up the local definitions, execute the list for the normal
% math version.
%    \begin{macrocode}
  \mv@normal
%    \end{macrocode}
%
% So now the offsets are all entered into the table, separated by
% |\or|.
% Finish off the definition by making this an |\ifcase|. Add a default
% value of zero, so that any symbol fonts declared later will also
% work, as long as a bold version is assigned to the bold math version.
%    \begin{macrocode}
  \expandafter\xdef\csname bm@#1table\endcsname{%
    \noexpand\ifcase\@tempcnta
      \csname bm@#1table\endcsname
    \noexpand\else
      \z@
    \noexpand\fi}%
%    \end{macrocode}
%
% Put things back as they were.
%    \begin{macrocode}
  \expandafter\split@name\bm@general\@nil
  \let\install@mathalphabet\bm@pmb
  \let\getanddefine@fonts\bm@pmb@
  \let\or\bm@pmb@@}
%    \end{macrocode}
% \end{macro}
% \end{macro}
% \end{macro}
%
% \begin{macro}{\bmmax}
% To save declaring too many symbol fonts, do not auto-declare
% any more than |\bmmax| bold symbol fonts into the normal math
% version. Any bold fonts not so allocated will be accessed via
% |\boldmath| which is slower and doesnt work for delimiters
% and accents. It may be set in the preamble with |\newcommand|
% but use |\chardef| here for a slight efficiency gain.
%
% If this is set to a higher value before this
% package is loaded, keep that value.
%    \begin{macrocode}
\ifx\bmmax\@undefined
  \chardef\bmmax=4
\fi
%    \end{macrocode}
%
% If there is no bold math version, It is very easy to set up
% the table, no need to use all the tricky code above.
% Also, at the end of the package redefine the internal macro
% that |\bm| uses to call |\boldmath|, to use poor man's bold
% instead.
%    \begin{macrocode}
\ifx\mv@bold\@undefined
  \def\bm@boldtable{\m@ne}
  \AtEndOfPackage{%
    \def\bm@gr@up#1#2{%
      \bm@pmb{#2}}}
\else
%    \end{macrocode}
% Otherwise use the definition of |\bm| above to set up |\bm@boldtable|
% by comparing the fonts available in the normal and bold math versions.
%    \begin{macrocode}
  \bm{bold}\bmmax
%    \end{macrocode}
%
% \begin{macro}{\mathbf}
% As the bold font has been defined as a symbol font, make |\mathbf|
% access that rather than have it allocate a new math group for the
% same font. (Just in case there were no free slots wrap this
% in an extra test.)
%    \begin{macrocode}
  \@ifundefined{symboldoperators}
    {}
    {\DeclareSymbolFontAlphabet\mathbf{boldoperators}}
%    \end{macrocode}
% \end{macro}
%
%    \begin{macrocode}
\fi
%    \end{macrocode}
% \end{macro}
%
% \begin{macro}{\hmmax}
%
% Same for heavy (but default to three this time (enough for mathtime
% plus, as no heavy operators font).
%    \begin{macrocode}
\ifx\hmmax\@undefined
  \chardef\hmmax=3
\fi
%    \end{macrocode}
%
% Similarly if there is a heavy math version, set up |\bm@heavytable|.
% (If there is no heavy math version, do nothing here, as |\hm| will be
% set to |\bm| later, once that is defined.)
%    \begin{macrocode}
\ifx\mv@heavy\@undefined
\else
  \bm{heavy}\hmmax
\fi
%    \end{macrocode}
% \end{macro}
%
%
% \begin{macro}{\bm@general}
% \changes{v1.0d}{1997/11/15}
%      {extra code to support prime lookahead and breqn}
% \changes{v1.0h}{2002/11/22}{Pass math version as third (new) argument}
%
% |\bm| is pretty much |\bmdefine\bm@command| followed by executing
% |\bm@command|. It would in principle be possible to execute the
% emboldened tokens directly, rather than building up a macro first,
% but (as I learned the hard way) it's difficult to do this in the midst
% of all these nested |\if| constructs.
% First extract the central bit of code for |\hm| |\bm| |\hmdefine| and
% |\bmdefine|. Note that in the case of the inline versions they take
% an argument and brace it, rather than relying on |\bm@general| to pick
% up the argument. This makes the code robust with respect to premature
% expansion.
%    \begin{macrocode}
\begingroup
\catcode`\'=\active
\@firstofone{\endgroup
\def\bm@general#1#2#3#4#5{%
  \begingroup
%    \end{macrocode}
% \changes{v1.0d}{1997/11/15}
%      {make nested \cs{bm} vanish completely, not leave a brace group}
% First locally disable |\bm| and |\hm|, as they would mess things
% up terribly, and the original Spivak versions used the syntax
% |\bmdefine\balpha{\bm\alpha}|.
%    \begin{macrocode}
    \let\bm\@firstofone
    \let\hm\@firstofone
%    \end{macrocode}
% Now initialise the commands used to save the tokens constructed.
%    \begin{macrocode}
    \global\let\bm@command\@empty
    \let\@let@token\@empty
%    \end{macrocode}
% As we want to expand the macros to look at their definition
% turn off protection. Otherwise the |\protect| will be carried over
% and apply to the wrong token, eg |{|.
%    \begin{macrocode}
    \let\protect\@empty
    \let\@typeset@protect\@empty
%    \end{macrocode}
% Set up either bold or heavy
%    \begin{macrocode}
    \def\bm@mathchoice{\bm@m@thchoice#1}%
    \def\bm@group{\bm@gr@up#1}%
    \let\bm@table#2%
%    \end{macrocode}
% Make sure |\left| and |\right| are really non expandable,
% and not |\ifx| equal to anything else.
% \changes{v1.0d}{1997/11/15}
%      {make sure \cs{left} is primitive}
%    \begin{macrocode}
    \let\left\holdinginserts
%    \end{macrocode}
% These three save on the number of |\ifx| tests below.
% \changes{v1.1a}{2003/09/01}{Forgotten to check for \cs{hskip} (pr/3572)}
%    \begin{macrocode}
    \let\right\left
    \let\mskip\mkern
    \let\hskip\kern
%    \end{macrocode}
% Definition of |'| locally modified so as not to use |\futurelet|
% in the look ahead, but to make the |\prime| available at the top level
% to be made bold, or heavy or whatever.
% |'| is locally active for this definition.
% \changes{v1.0d}{1997/11/15}
%      {prime code added}
%    \begin{macrocode}
    \let\bm@prime\copy
    \def'{\bm@prime\prime\relax}%
%    \end{macrocode}
% For optional argument commands. This expandable version of
% |\@ifnextchar| is not 100\% safe, but works for |\sqrt| unless
% you put something really strange in the arguments.
% \changes{v0.11}{1997/01/06}
%      {\cs{@ifnextchar} made safe.}
%    \begin{macrocode}
    \def\@ifnextchar##1##2##3##4{%
      \if##1##4%
        \expandafter\@firstoftwo
      \else
        \expandafter\@secondoftwo
      \fi
      {##2##4}{##3{##4}}}%
%    \end{macrocode}
% For Vladimir Volovich\ldots
% \changes{v1.0f}{1998/12/10}
%      {Errors and warnings made safe. tools/2917}
%    \begin{macrocode}
    \def\GenericWarning##1##2{%
      \unvcopy{\GenericWarning{##1}{##2}}}%
    \def\GenericError##1##2##3##4{%
      \unvcopy{\GenericError{##1}{##2}{##3}{##4}}}%
%    \end{macrocode}
% For AMS definitions.
%    \begin{macrocode}
    \let\DN@\copy
    \let\FN@\copy
    \let\next@\copy
    \global\let\bm@first\@empty
%    \end{macrocode}
%   For AMS version of |\sqrt|: don't expand just wrap in brace group
%   so that it can be made bold in a safe but slow way. Do the same for
%   internal accent command
% \changes{v1.1b}{2003/10/05}{AMS \cs{sqrt} not working}
%
% \changes{v1.1c}{2004/02/26}{\cs{accentV} made safe (pr/3625)}
% Code for AMS accent allows bm to be used (just) with accent
% but stops the nested accents stacking correctly, this can be
% corrected by using an extra brace group as usual.
% |\bm{{\hat{\hat{F}}}}|
%    \begin{macrocode}
    \ifx\uproot@\undefined\else
       \def\root##1\of##2{{\root##1\of{##2}}}%
    \fi
    \def\mathaccentV##1{\mathaccent"\accentclass@}%
%    \end{macrocode}
% For breqn definitions.
%    \begin{macrocode}
    \let\@ifnext\@ifnextchar
    \let\measure@lhs\copy
    \let \rel@break\copy
    \let \bin@break\copy
    \let \after@open\copy
    \let \after@close\copy
%    \end{macrocode}
% Make sure things like |\pounds| take the `math branch' even
% in |\bmdefine| (which is not executed in math mode).
%    \begin{macrocode}
    \let\ifmmode\iftrue
%    \end{macrocode}
%    We have to ensure that the math alphabets have definitions that
%    correspond the the ``bold'' math version we are going to switch
%    to. As these definitions are globally assigned when a math
%    version is changed it is likely that right now we have those of
%    the normal math version active. Argument |#3| holds either
%    |\mv@bold| or |\mv@heavy| and we execute that after redefining
%    |\install@mathalphabet| and |\getanddefine@fonts| suitably.
%    The definitions are reverted back to their original the moment
%    the scanning is done
% \changes{v1.0h}{2002/11/22}{Get math alphabets right (pr/3476)}
%    \begin{macrocode}
     \let\install@mathalphabet\def
     \let\getanddefine@fonts\@gobbletwo
     #3%
%    \end{macrocode}
% The last redefinition just makes |\mathit| type commands re-insert
% themselves (more or less) as if they are allowed to expand
% they die horribly if the expansions are put into |\mathchoice|
% and so executed more than once.
%    \begin{macrocode}
    \def\select@group##1##2##3##4{{%
      \protect##1{##4}}}%
    \def\use@mathgroup##1##2##3{{%
      \protect\use@mathgroup##1{##2}{##3}}}%
%    \end{macrocode}
%
% So now start looking at the argument.
% \changes{v0.10}{1997/01/04}
%      {Use \cs{bm@end}}
%    \begin{macrocode}
    \bm@expand#5\bm@end
  \endgroup
%    \end{macrocode}
%
% Finally outside the group either execute |\bm@command| (for |\bm|)
% or save its definition (for |\bmdefine|).
%    \begin{macrocode}
  #4}
%    \end{macrocode}
% End of the |\@firstofone| above, and the scope of the active |'|.
%    \begin{macrocode}
}
%    \end{macrocode}
% \end{macro}
%
% \begin{macro}{\bm}
% \changes{v1.0d}{1997/11/15}
%      {Make \cs{bm} grab its argument even when protected}
% Set up the bold (rather than heavy) version, and run |\bm@command|
% right at the end, to execute the emboldened argument.
% The argument is grabbed by the top level function, and explicitly
% braced, so that |\bm| works even if the braces are omitted round
% its argument in a `moving argument'.
% \changes{v1.0h}{2002/11/22}{Pass math version as third (new) argument}
%    \begin{macrocode}
\DeclareRobustCommand\bm{%
  \bm@general\boldmath\bm@boldtable\mv@bold\bm@command}
\protected@edef\bm#1{\bm{#1}}
%    \end{macrocode}
% \end{macro}
%
% \begin{macro}{\DeclareBoldMathCommand}
% \changes{v0.07}{1996/12/14}
%      {Macro added}
% \begin{macro}{\bm@declare}
% \changes{v0.07}{1996/12/14}
%      {Macro added}
% |DeclareBoldMathCommand|%^^A
%      \oarg{mathversion}\marg{command}\marg{math expression}\\
% looks like |\bm| except at the end the specified command is
% globally defined to be |\bm@command|.
% The \meta{mathversion} defaults to `bold'.
%    \begin{macrocode}
\def\DeclareBoldMathCommand{\@testopt\bm@declare{bold}}
%    \end{macrocode}
%
% \changes{v1.0h}{2002/11/22}{Pass math version as third (new) argument}
%    \begin{macrocode}
\def\bm@declare[#1]#2{%
  \expandafter\bm@general
        \csname #1math\expandafter\endcsname
        \csname bm@#1table\expandafter\endcsname
        \csname mv@#1\endcsname
        {\bm@define#2}}
%    \end{macrocode}
% \end{macro}
% \end{macro}
%
% \begin{macro}{\bmdefine}
% |\bmdefine| Shorthand for |\DeclareBoldMathCommand[bold]|.
%
% |\bm| is empty within the definition, so that either\\
% |\bmdefine\balpha{\bm\alpha}| or  |\bmdefine\balpha{\alpha}| \\
% may be used. (The former just for compatibility with the original
% version for plain \TeX).
%    \begin{macrocode}
\def\bmdefine{\DeclareBoldMathCommand[bold]}
%    \end{macrocode}
% \end{macro}
%
% \begin{macro}{\hm}
% \changes{v1.0d}{1997/11/15}
%      {Make \cs{hm} grab its argument even when protected}
% \begin{macro}{\hmdefine}
% Same again for |\hm|.
%    \begin{macrocode}
\ifx\mv@heavy\@undefined
%    \end{macrocode}
%
% If there is no heavy math version defined, let |\hm| be defined
% as |\bm|. Currently there is no warning given, perhaps there should
% be, or even an error?
%    \begin{macrocode}
  \let\hm\bm
  \let\heavymath\boldmath
  \let\bm@heavytable\bm@boldtable
%    \end{macrocode}
%
%    \begin{macrocode}
\else
%    \end{macrocode}
%
% Otherwise define |\hm| and |\hmdefine| in direct analogy with the
% above.
% \changes{v1.0h}{2002/11/22}{Pass math version as third (new) argument}
%    \begin{macrocode}
  \DeclareRobustCommand\hm{%
    \bm@general\heavymath\bm@heavytable\mv@heavy\bm@command}
  \protected@edef\hm#1{\hm{#1}}
%    \end{macrocode}
%
%    \begin{macrocode}
  \def\hmdefine{\DeclareBoldMathCommand[heavy]}
%    \end{macrocode}
%
%    \begin{macrocode}
\fi
%    \end{macrocode}
% \end{macro}
% \end{macro}
%
% \begin{macro}{\bm@end}
% \changes{v0.10}{1997/01/04}
%      {Macro added}
% Normally speaking |\outer| declarations should be avoided at all
% costs. (\LaTeX\ redefines all of plain \TeX's allocation macros
% to be non-outer.) However this is one place where it seems like a
% good idea. If a command taking an argument is put in |\bm| without
% its argument, then the |\@@end| terminating token would be taken as
%  the argument, and so the rest of the paragraph would be gobbled up
% and the \LaTeX\ would die horribly. So make the internal terminating
% token |\outer|. (The actual test for termination is made against
% |\@@end| not |\bm@end| as this macro will be expanded by the look-ahead
% system.)
%    \begin{macrocode}
\outer\def\bm@end{\@@end}
%    \end{macrocode}
% \end{macro}
%
% \begin{macro}{\bm@expand}
% \changes{v0.11}{1997/01/06}
%      {Double up.}
% \begin{macro}{\bm@exp@nd}
% \changes{v0.11}{1997/01/06}
%      {Macro added}
% |\afterassignment| trick to fully expand the following tokens until
% the first non-expandable token is revealed. This may discard a space
% token (which is what \TeX\ is looking for) but that doesn't matter in
% math mode. The expansion lookahead is done twice in case any stray
% space tokens have crept in.\footnote{The need for this was noticed
% while testing \cs{sqrt}. The definition of \cs{root} inherited from
% plain \TeX\ has an anomalous space token, that is normally harmless
% (just wastes memory), but which killed earlier versions of this
%  package.}
%    \begin{macrocode}
\def\bm@expand{\afterassignment\bm@exp@nd\count@`\a}
%    \end{macrocode}
%
%    \begin{macrocode}
\def\bm@exp@nd{\afterassignment\bm@test\count@`\a}
%    \end{macrocode}
% \end{macro}
% \end{macro}
%
% \begin{macro}{\bm@test}
% Normally we will grab the non-expandable token as a macro argument
% but better check it is not |{| first. Save the previous token
% so we can check later if it was |\left|, in which case use the delcode
% rather than the mathcode if the current token is a character.
%    \begin{macrocode}
\def\bm@test{%
  \let\bm@previous\@let@token
  \futurelet\@let@token\bm@test@}
%    \end{macrocode}
% \end{macro}
%
% \begin{macro}{\bm@test@}
% If looking at a single token, switch to |\bm@test@token|, else if
% looking at a |{ }| group, grab the whole group with |\bm@group|.
% A |\bgroup| token will take the wrong branch here (currently not
% trapped).
%    \begin{macrocode}
\def\bm@test@{%
  \ifx\@let@token\bgroup
    \expandafter\bm@group
  \else
    \expandafter\bm@test@token
  \fi}
%    \end{macrocode}
% \end{macro}
%
% \begin{macro}{\bm@gr@up}
% \changes{v1.0c}{1997/10/09}
%      {Extra brace around argument for \cs{over}}
% If faced with a group,
% If we are in math mode, stick it in a |\boldsymbol| like construct
% and then recurse on |\bm@expand|.
% Otherwise just use |\bfseries\boldmath|.
% The actual test is deferred till `run time'.
% Here and elsewhere could deal with the inner list with an inner call
% to |\bm|, but that doesn't seem to gain very much, and complicates the
% code quite a bit.
%
% |#1| is either |\boldmath| or |\heavymath|.
% Need to add an extra set of explicit braces around |#2| as otherwise
% the math style commands applied in |\mathchoice| might only apply
% to the first half of an |\over| construction.
%    \begin{macrocode}
\def\bm@gr@up#1#2{%
  \bm@add{{\bm@gr@@p#1{{#2}}}}}
%    \end{macrocode}
% \end{macro}
%
% \begin{macro}{\bm@gr@@p}
% |#1| is either |\boldmath| or |\heavymath|.
%    \begin{macrocode}
\def\bm@gr@@p#1#2{%
  \ifmmode
    \bm@mchoice#1{#2}{#2}{#2}{#2}%
  \else
    \bfseries#1#2%
  \fi}
%    \end{macrocode}
% \end{macro}
%
% \begin{macro}{\bm@test@token}
% If not facing a |{  }| group then test to see what we have.
% Basic idea: Trap |\mathchardef| tokens, character tokens, and calls to
% |\mathchar|, |\mathaccent|, etc, and change the \emph{math-group}
% (fam) to point at the bold version.
% Other things just copy straight over to the command being built.
% (Anything inside a |\mathop| or similar will end up being made bold
% as the |\mathop| will be copied over, but its argument will be made
% bold by the group code above.
%    \begin{macrocode}
\def\bm@test@token#1{%
  \let\bm@next\@empty
%    \end{macrocode}
%
% Stop here. Note that it is vital that the terminating token
% is non-expandable and defined, rather than the usual \LaTeX\
% terminators |\@nil| or |\@@|. (Worse still would be a `quark'
% like |docstrip|'s |\qStop|.)
%    \begin{macrocode}
  \ifx#1\@@end
%    \end{macrocode}
%
% |\bm@mathchoice| uses macro arguments, so need to make the tail
% recursion explicit here. All the other cases recurse by way of
% |\afterassignment| which means all the trailing |\fi| are eaten
% while making the assignment.
%    \begin{macrocode}
  \else\ifx#1\mathchoice
    \let\bm@next\bm@mathchoice
%    \end{macrocode}
%
% The main point: Find these expressions, and change the mathgroup.
%    \begin{macrocode}
  \else\ifx#1\mathchar
    \afterassignment\bm@mathchar\count@
  \else\ifx#1\mathaccent
    \afterassignment\bm@mathaccent\count@
  \else\ifx#1\delimiter
    \afterassignment\bm@delimiter\count@
  \else\ifx#1\radical
    \afterassignment\bm@radical\count@
%    \end{macrocode}
%
% Need to trap spaces otherwise digits will get turned to bold
% mathchars.
% \changes{v1.1a}{2003/09/01}{Forgotten to check for \cs{hskip} (pr/3572)}
%    \begin{macrocode}
  \else\ifx#1\mkern
    \bm@register#1{\muskip\z@}%
  \else\ifx#1\kern
    \bm@register#1\skip@
  \else\ifx#1\penalty
    \bm@register#1\count@
%    \end{macrocode}
%
% \changes{v1.0f}{1998/12/10}
%      {Add possibility to copy brace group unmodified. tools/2917}
% |\vcopy| is a flag to copy the next group unchanged to the
% result command.
%    \begin{macrocode}
  \else\ifx#1\unvcopy
    \let\bm@next\bm@add
%    \end{macrocode}
%
%    \begin{macrocode}
  \else\ifcat\noexpand#1\relax
%    \end{macrocode}
% Other command, look if it's a mathchardef token (otherwise just add
% it).
%    \begin{macrocode}
    \xdef\meaning@{\meaning#1}%
    \expandafter\bm@mchar@test\meaning@""\@nil#1%
%    \end{macrocode}
% \changes{v1.0d}{1997/11/15}
%      {New active \cs{mathcode} code.}
% Character token.
% If it is of catcode 11 or 12, get its mathcode.
% If that is |"8000| replace the token by its active version, and then
% let bm expansion look again at the character. Being really active
% this time, it will expand away (probably).
%
% If the previous token was |\left| or |\right|, get the delcode
% instead of the mathcode.
%    \begin{macrocode}
  \else\ifcat.\ifcat a#1.\else#1\fi
    \count@\mathcode`#1\relax
    \ifnum\count@=\mathcode`\'%
      \begingroup\uccode`\~`#1\uppercase{\endgroup
        \def\bm@next{\bm@expand~}}%
    \else
      \ifx\bm@previous\left
        \count@\delcode`#1\relax
        \bm@delimiter
      \else
        \bm@mathchar
      \fi
    \fi
  \else
%    \end{macrocode}
% And final possibility:
% a character token of catcode other than 11 or 12.
%    \begin{macrocode}
    \bm@add{#1}%
  \fi\fi\fi\fi\fi\fi\fi\fi\fi\fi\fi\fi
  \bm@next}
%    \end{macrocode}
% \end{macro}
%
%
% \begin{macro}{\bm@define}
% \changes{v0.08}{1996/12/18}
%      {Defined commands now always robust}
% End code for |\bmdefine|. Define the given command name to the
% robust form of the accumulated code.
%
% If |\bm@command| is equal to |\@gtempa| then it is a macro whose
% expansion is a single call to |\mathchar|, so that can be optimised
% with a |\mathchardef|.
%    \begin{macrocode}
\def\bm@define#1{%
  \begingroup
    \ifx\bm@command\@gtempa
      \def\mathchar{\global\mathchardef#1}%
      \bm@command
    \else
%    \end{macrocode}
% Rather than simply |\let#1\bm@command|, make the defined command
% robust. |\bm@first| is normally empty, but might be something like
% |\DOTSI| which needs to be lifted to the top level, in front
% of any |\protect| because of the lookahead mechanism used
% in the \textsf{amsmath} package.
%    \begin{macrocode}
      \toks@\expandafter{\bm@command}%
      \xdef#1{\bm@first\noexpand\bm@protect\noexpand#1{\the\toks@}}%
    \fi
  \endgroup}
%    \end{macrocode}
% \end{macro}
%
%
% \begin{macro}{\bm@protect}
% \changes{v0.08}{1996/12/18}
%      {macro added}
% Commands defined by |\bmdefine| re-insert themselves
% if protection is enabled.
%    \begin{macrocode}
\def\bm@protect#1{%
  \ifx\protect\@typeset@protect
    \expandafter\@firstofone
  \else
    \protect#1\expandafter\@gobble
  \fi}
%    \end{macrocode}
% \end{macro}
%
% \begin{macro}{\bm@mchoice}
% |\boldsymbol|, more or less.
% |#1| is either |\boldmath| or |\heavymath|.
%    \begin{macrocode}
\def\bm@mchoice#1#2#3#4#5{%
  \mathchoice{\hbox{#1$\displaystyle\m@th#2$}}%
             {\hbox{#1$\textstyle\m@th#3$}}%
             {\hbox{#1$\scriptstyle\m@th#4$}}%
             {\hbox{#1$\scriptscriptstyle\m@th#5$}}}
%    \end{macrocode}
% \end{macro}
%
% \begin{macro}{\bm@m@thchoice}
% Action if you find a |\mathchoice|. Add the bold version to
% |\bm@command| then recurse.
%
% |#1| is either |\boldmath| or |\heavymath|.
%    \begin{macrocode}
\def\bm@m@thchoice#1#2#3#4#5{%
  \bm@add{\bm@mchoice#1{#2}{#3}{#4}{#5}}}
%    \end{macrocode}
% \end{macro}
%
% \begin{macro}{\bm@register}
% \changes{v0.07}{1996/12/14}
%      {Macro added}
% Combined code for setting up |\bm@r@gister| with the correct
% register type.
%    \begin{macrocode}
\def\bm@register#1#2{%
  \def\@tempa{#1\the#2}%
  \afterassignment\bm@r@gister#2}
%    \end{macrocode}
% \end{macro}
%
% \begin{macro}{\bm@r@gister}
% \changes{v0.06}{1996/12/12}
%      {Support \cs{mskip}}
% \changes{v0.07}{1996/12/14}
%      {Combine all register macros}
% |\mkern| itself would transfer to |\bm@command| without any special
% test, but any explict dimension following would be converted to
% |\mathchar|. So trap this and grab the muskip as a muskip.
% This is used in |\iiint|. |\penalty| was needed for the
% AMS  version of |\colon|, and so do most of the others as well.
%    \begin{macrocode}
\def\bm@r@gister{%
  \bm@xadd{\@tempa\space}}
%    \end{macrocode}
% \end{macro}
%
%
% \begin{macro}{\bm@mathchar}
% \changes{v0.10}{1997/01/04}
%      {Modify to use \cs{boldmath} or \cs{pmb} if needed.}
% Change the family (math group) of a mathcode and then
% use the modified code with |\mathchar|. If there is no
% suitable bold font in the current math version, use the original
% unmodified mathcode, but switch to |\boldmath| (if there is a bold
% font there) or use `poor man's bold'. Note that these other
% possibilities are only possible here, not for the otherwise similar
% code for |\delimiter| or |\mathaccent|, as those commands
% must work with fonts from the same math version.
%
% Finally recurse down the list.
%    \begin{macrocode}
\def\bm@mathchar{%
%    \end{macrocode}
%
%    \begin{macrocode}
  \@tempcntb\count@
  \let\@tempa\bm@group
%    \end{macrocode}
%
% |\bm@changefam| will isolate the math group from the mathcode
% and look up the offset in the current table.
%    \begin{macrocode}
  \bm@changefam{}%
%    \end{macrocode}
%
% If the mathcode has changed, then just add the new |\mathchar|
% (saving |\@gtempa| allows |\bmdefine| to optimise this to a
%  mathchardef if it turns out to be the only symbol in the argument).
%    \begin{macrocode}
  \ifnum\count@>\@tempcntb
%    \end{macrocode}
%
%    \begin{macrocode}
    \ifx\bm@command\@empty
      \xdef\@gtempa{\mathchar\the\count@\space}%
    \fi
    \bm@xadd{\mathchar\the\count@\space}%
  \else
%    \end{macrocode}
%
% Otherwise grab the math class from the math code
% and add that (locally zapping |\bm@expand| as we don't want
% to recurse at this point).
%    \begin{macrocode}
    \begingroup
      \divide\count@"1000
      \let\bm@expand\relax
      \bm@xadd\bm@class
    \endgroup
%    \end{macrocode}
% |\@tempa| will be |\bm@group| (which applies |\boldmath| and
% |\mathchoice|) unless it was changed by |\bm@changefam| to |\bm@pmb|
% (which applies a `poor man's bold' construction in a |\mathchoice|).
%    \begin{macrocode}
    \edef\@tempb{%
      \noexpand\@tempa{\mathchar\the\count@\space}}%
    \@tempb
  \fi}
%    \end{macrocode}
% \end{macro}
%
% \begin{macro}{\bm@pmb}
% \changes{v0.10}{1997/01/04}
%      {Macro added}
% Add a poor man's bold construction to the list being built.
%    \begin{macrocode}
\def\bm@pmb#1{%
  \bm@add{\bm@pmb@{#1}}}
%    \end{macrocode}
% \end{macro}
%
% \begin{macro}{\bm@pmb@}
% \changes{v0.10}{1997/01/04}
%      {Macro added}
% |\pmb| variant. (See \TeX{}Book, or AMS \textsf{amsbsy} package).
% This one takes a bit more care to use smaller offsets in subscripts.
%    \begin{macrocode}
\def\bm@pmb@#1{{%
  \setbox\tw@\hbox{$\m@th\mkern.4mu$}%
  \mathchoice
    \bm@pmb@@\displaystyle\@empty{#1}%
    \bm@pmb@@\textstyle\@empty{#1}%
    \bm@pmb@@\scriptstyle\defaultscriptratio{#1}%
    \bm@pmb@@\scriptscriptstyle\defaultscriptscriptratio{#1}}}
%    \end{macrocode}
% \end{macro}
%
% \begin{macro}{\bm@pmb@@}
% \changes{v0.10}{1997/01/04}
%      {Macro added}
% Helper macro. Box |#3| and set it three times in the style |#1|,
% offset by an amount reduced by the ratio specified in |#2|.
%    \begin{macrocode}
\def\bm@pmb@@#1#2#3{{%
  \setbox\z@\hbox{$\m@th#1#3$}%
  \dimen@#2\wd\tw@
  \rlap{\copy\z@}%
  \kern\dimen@
  \raise1.5\dimen@\rlap{\copy\z@}%
  \kern\dimen@
  \box\z@}}%
%    \end{macrocode}
% \end{macro}
%
% \begin{macro}{\bm@class}
% \changes{v0.10}{1997/01/04}
%      {Macro added}
% Convert a numeric math class back to a math class command.
% |\mathord| is omitted in class $0$ and $7$ to save space and so
% things work out right in constructions such as |x^a| where
% |x^\mathord{a}| would not work.
%    \begin{macrocode}
\def\bm@class{%
  \ifcase\count@
    \or
    \mathop\or
    \mathbin\or
    \mathrel\or
    \mathopen\or
    \mathclose\or
    \mathpunct\or
  \fi}
%    \end{macrocode}
% \end{macro}
%
% \begin{macro}{\bm@add}
% \changes{v0.07}{1996/12/14}
%      {macro added}
% A version of |\g@addto@macro| that internally uses a |\begingroup|
% rather than a brace group\footnote{This bug is fixed in the \LaTeX\
% kernel of 1996/12/01}, to save creating a mathord.
%
% As need to redefine it anyway, save some tokens by making
% it specific to |\bm@command|, and to execute |\bm@expand|
% to continue the loop.
%    \begin{macrocode}
\def\bm@add#1{%
  \begingroup
    \toks@\expandafter{\bm@command#1}%
    \xdef\bm@command{\the\toks@}%
  \endgroup
  \bm@expand}
%    \end{macrocode}
% \end{macro}
%
% \begin{macro}{\bm@xadd}
% An |\xdef| version of |\bm@add|.
%    \begin{macrocode}
\def\bm@xadd#1{%
  \begingroup
    \toks@\expandafter{\bm@command}%
    \xdef\bm@command{\the\toks@#1}%
  \endgroup
  \bm@expand}
%    \end{macrocode}
% \end{macro}
%
% \begin{macro}{\bm@mathaccent}
% \changes{v1.0e}{1997/11/21}
%      {Use bm on argument group, as spotted on c.t.t.}
% |\mathaccent| version of |\bm@mathchar|.
%    \begin{macrocode}
\def\bm@mathaccent{%
 \bm@changefam{}%
%    \end{macrocode}
% The next four lines were added a v1.0e. Without them |\bm{\hat{A}}|
% makes the accent bold using |\bm| but the group |{A}| is made bold
% via a |\mathchoice| construction as for any other group, as |\bm|
% does not attempt to parse inside brace groups. While that produces
% something acceptable for lower case letters, it produces
% $\bm{\hat{{A}}}$ which is not too good. The braces may simply be
% omitted:
% |\bm{\hat A}| would work, producing $\bm{\hat A}$, however I did not
% want to document such a restriction, so now modify bm so that such
% brace groups are handled gracefully.
%
% It would be possible to locally make mathaccents take an argument
% during the bm look-ahead, so the brace groups would then vanish
% during expansion, however I would then need to explitly skip past
% \meta{filler} and also make sure that the end of parse token
% was not gobbled in marginal cases like |$\bm\hat$|.
%
% So instead do the following which gets rid of \meta{filler}
% with a redefinition of |\relax|, and just locally changes
% |\bm@group| so that instead of doing a |\mathchoice| it simply adds
% |\bgroup| and |\egroup| around the tokens, and lets bm modify the
% tokens of the `argument'. This means that |\bm{\hat{A}}| now produces
%\begin{verbatim}
% \mathaccent 29790 \bgroup \mathchar 30017 \egroup
%\end{verbatim}
% The inner math list is a single mathchar, and so \TeX\ will not box
% it, and the math accent will correctly position, taking into account
% the skewchar information.
%
% As the normal bm lookahead is used, it is automatic that the parse
% will end without trying to go past |\bm@end|.
%
% One disadvantage is that the group will mean that |\bm@previous|
% will not be correctly updated. However that is only used for
% delimiter checking, so can not matter here.
%    \begin{macrocode}
 \begingroup
 \def\bm@group##1{\endgroup\bm@xadd{\bgroup}##1\egroup}%
 \def\bm@test@token{\endgroup\bm@test@token}%
 \let\relax\@empty
%    \end{macrocode}
%
%    \begin{macrocode}
 \bm@xadd{\mathaccent\the\count@\space}}
%    \end{macrocode}
% \end{macro}
%
% \begin{macro}{\bm@delimiter}
% \changes{v1.0d}{1997/11/15}
%      {Support \cs{left}. null delimiter.}
% Change both families (math groups) of a delcode and then
% use the modified code with  |\delimiter|. Don't change code `0'
% as that denotes a null delimiter.
%    \begin{macrocode}
\def\bm@delimiter{%
  \ifnum\count@>\z@
    \bm@changefam{}%
    \bm@changefam{000}%
  \fi
  \bm@xadd{\delimiter\the\count@\space}}%
%    \end{macrocode}
% \end{macro}
%
% \begin{macro}{\bm@radical}
% Same for |\radical|.
%    \begin{macrocode}
\def\bm@radical{%
 \bm@changefam{}%
 \bm@changefam{000}%
 \bm@xadd{\radical\the\count@\space}}%
%    \end{macrocode}
% \end{macro}
%
% \begin{macro}{\bm@mchar@}
% Catcode 12 |\mathchar|, for |\ifx| tests.
%    \begin{macrocode}
\edef\bm@mchar@{\meaning\mathchar}
%    \end{macrocode}
% \end{macro}
%
% \begin{macro}{\bm@mchar@test}
% Test if the |\meaning| starts with |\mathchar|. If it does, grab the
% value into |\count@| and call |\bm@mathchar|, else just copy the
% command into the accumulated tokens.
% |#1|, |#2|, |#3| are all |\meaning| produced tokens, or `dummy tokens'
% added at the time this is called. |#4| is the original token, in case
%  decide not to use the |\meaning|.
%    \begin{macrocode}
\def\bm@mchar@test#1"#2"#3\@nil#4{%
  \xdef\meaning@{#1}%
  \ifx\meaning@\bm@mchar@
    \count@"#2\relax
    \bm@mathchar
  \else
%    \end{macrocode}
% Some other command: copy it straight over. If it is the first thing
% added, and it is a |\relax| token, save it in |\bm@first| for use
% in |\bm@define|.
%    \begin{macrocode}
    \ifx\bm@previous\@empty
      \ifx\relax#4%
        \gdef\bm@first{#4}%
      \fi
    \fi
    \bm@add{#4}%
  \fi}
%    \end{macrocode}
% \end{macro}
%
% \begin{macro}{\bm@changefam}
% \changes{v0.10}{1997/01/04}
%      {Rewrite for new \cs{bm@table} system}
% Pull out one specified hex digit and passes
% it to |\bm@modify| to change. argument is empty normally but |000| to
% access the second math group in a delimiter code.
%    \begin{macrocode}
\def\bm@changefam#1{%
  \@tempcnta\count@
  \divide\@tempcnta"1000#1 %
  \multiply\@tempcnta"1000#1 %
  \advance\@tempcnta-\count@
  \divide\@tempcnta-"100#1 %
%    \end{macrocode}
% Having isolated the required math group (fam), look up
% the offset in the current table.
%    \begin{macrocode}
  \@tempcnta\bm@table
%    \end{macrocode}
% If the offset is $-1$, keep |\count@| unchanged, but set |\@tempa|
% to use poor man's bold. Otherwise increment |\count@| to change the
% math group specified.
%    \begin{macrocode}
  \ifnum\@tempcnta=\m@ne
    \let\@tempa\bm@pmb
  \else
    \multiply\@tempcnta"100#1 %
    \advance\count@\@tempcnta
  \fi}
%    \end{macrocode}
% \end{macro}
%
% \begin{macro}{\bm@prime}
% \changes{v1.0d}{1997/11/15} 
%      {Macro added}  
% Support |'|. Earlier versions did not make the prime bold in a'.
%
% |\bm{a''}| will now produce (with the normal encodings)
%\begin{verbatim}
%      \mathchar 30049 
%      \bm@prime \mathchar 1584 \relax
%      \bm@prime \mathchar 1584 \relax
%\end{verbatim}
% So |\bm@prime| does essentially the same as the active definition of
% |'|, which is to start a superscript group then keep adding |\prime|
% for each |'| (or |\bm@prime|) following. Here modified to grab a
% |\relax| delimited argument and use that instead of |\prime|.
% |\bm@prime| is locally |\let| to |'| so the |\ifx| tests in |\pr@m@s|
% don't need changing.
%    \begin{macrocode}
\def\bm@prime{^\bgroup
  \let\bm@prime'%
  \def\prim@s##1\relax{##1\futurelet\@let@token\pr@m@s}%
  \prim@s}
%    \end{macrocode}
% \end{macro}
%
% \begin{macro}{\boldsymbol}
% \changes{v1.0b}{1997/04/14}
%      {Macro added}
% \begin{macro}{\heavysymbol}
% Finally, to ease conversion of documents between this package and
% the \textsf{amsbsy} package:
%    \begin{macrocode}
\let\boldsymbol\bm
%    \end{macrocode}
%
%    \begin{macrocode}
\let\heavysymbol\hm
%    \end{macrocode}
% \end{macro}
% \end{macro}
%    \begin{macrocode}
%</package>
%    \end{macrocode}
%
% \Finale
%
